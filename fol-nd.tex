\section{Dedução Natural}
Como visto na seção anterior, devemos estabelecer as regras de inclusão e exclusão
de dedução para os quantificadores e para a igualdade. Listamos elas inicialmente:
\newline \textbf{Quantificador universal:}
 \newline 
 \begin{center}
 \begin{bprooftree}
    \AxiomC{$A(y)$}
    \RightLabel{\scriptsize $\forall I$}
    \UnaryInfC{$\forall x A(x)$}
\end{bprooftree}
\begin{bprooftree}
    \AxiomC{$\forall x A(x)$}
    \RightLabel{\scriptsize $\forall E$}
    \UnaryInfC{$A(t)$} 
\end{bprooftree}
\end{center}

Para inserir o quantificador universal devemos ter que a variável $y$ não deve estar 
atrelada em nenhuma hipótese, isto é, em todas as hipóteses não canceladasela não pode ser livre. 
\newline \textbf{Quantificador existencial:}

 \begin{center}
    \begin{bprooftree}
        \AxiomC{$A(t)$}
        \RightLabel{\scriptsize $\exists I$}
        \UnaryInfC{$\exists x A(x)$}
    \end{bprooftree}
    \begin{bprooftree}
        \AxiomC{$\exists x A(x)$}
        \AxiomC{}
        \UnaryInfC{$A(t)$}
        \alwaysNoLine
        \UnaryInfC{$\vdots$}
        \UnaryInfC{$B$}
        \alwaysSingleLine
        \RightLabel{\scriptsize $\exists E$}
        \BinaryInfC{$B$}
    \end{bprooftree}
 \end{center}

Para retirarmmos o quantificador existencial, a variável $t$ não pode estar livre em $B$, isto é, ela não 
deve estar livre em qualquer hipótese não cancelada.

\textbf{Igualdade:}
\begin{center}
    \begin{bprooftree}
        \AxiomC{}
        \RightLabel{\scriptsize refl}
        \UnaryInfC{$t=t$}
    \end{bprooftree}
    \begin{bprooftree}
        \AxiomC{$t=s$}
        \RightLabel{\scriptsize sim}
        \UnaryInfC{$s=t$} 
    \end{bprooftree}
\end{center}
\begin{center}
    \begin{bprooftree}
        \AxiomC{$t=s$}
        \AxiomC{$s=v$}
        \RightLabel{\scriptsize trans}
        \BinaryInfC{$t=v$}  
    \end{bprooftree}
    \begin{bprooftree}
        \AxiomC{$t=s$}
        \RightLabel{\scriptsize subs}
        \UnaryInfC{$r(t) = r(s)$}
    \end{bprooftree}
    \begin{bprooftree}
        \AxiomC{$t=s$}
        \AxiomC{$P(t)$}
        \RightLabel{\scriptsize subs}
        \BinaryInfC{$P(s)$}
    \end{bprooftree}
\end{center}
Iremos passar por cada um destas regras para apresentá-las com o auxílio de exemplos.

\subsection{Quantificador universal}
Como primeiro exemplo segue abaixo uma dedução natural de 
$(\forall x P(x) \to \forall x Q(x)) \to \forall x (P ( x) \to Q (x))$. 
Note que apesar de parecer uma implicação de duas fórmulas idênticas, na premissa a propriedade
 $P$ recebe uma variável e a propriedade $Q$ pode receber outra variável, enquanto que na conclusão
  o valor que $x$ assume é o mesmo para $P$ e $Q$.

\begin{center}
    \AxiomC{}
    \RightLabel{\scriptsize 1}
    \UnaryInfC{$\forall x P(x) \to \forall x Q(x)$}
    \RightLabel{\scriptsize $\forall E$}
    \UnaryInfC{$P(t) \to \forall x Q(x)$}
    \RightLabel{\scriptsize $\forall E$}
    \UnaryInfC{$P(t) \to Q(t)$}
    \RightLabel{\scriptsize $\forall I$}
    \UnaryInfC{$\forall x (P(x) \to Q(x))$}
    \RightLabel{\scriptsize 1}
    \UnaryInfC{$\forall x P(x) \to \forall x Q(x) \to \forall x (P(x) \to Q(x))$}
    \DisplayProof
\end{center}

O primeiro passo será a exclusão da implicação (a implicação principal da fórmula), 
assumindo $\forall x P(x) \to \forall x Q(x)$ como nossa hipótese. Aplicamos uma primeira exclusão do
 universal em $\forall x P(x)$ e novamente aplicamos a exclusão do universal em $\forall x Q(x)$. Note que 
 ao definirmos a variável com o mesmo nome $t$ em ambas as exclusões do universal foi o que permitiu inserir 
 um universal que englobe ambas variáveis, que é o passa seguinte.
\newline Vamos para um outro exemplo, utilizando as hipóteses 
$\forall x (\neg Q(x) \to R(x))$ e $\forall x(P(x) \land \neg Q(x))$ provaremos $\forall x R(x)$:

\begin{center}
    \AxiomC{$\forall x(P(x) \land \neg Q(x))$}
    \RightLabel{\scriptsize $\forall E$}
    \UnaryInfC{$P(t) \land \neg Q(t)$}
    \UnaryInfC{$\neg Q(t)$}
    \AxiomC{$\forall x (\neg Q(x) \to R(x))$}
    \RightLabel{\scriptsize $\forall E$}
    \UnaryInfC{$\neg Q(t) \to R(t)$}
    \BinaryInfC{$R(t)$}
    \RightLabel{\scriptsize $\forall I$}
    \UnaryInfC{$\forall x R(x)$}
    \DisplayProof
\end{center}
Vamos inicialmente utilizar de nossas hipóteses, em $\forall x(P(x) \land \neg Q(x))$ nós executamos a primeira 
exclusão do universal, utilizando a variável $t$ e em seguida executamos a exclusão do $\land$ 
visto que não é necessário o $P(t)$ na dedução. Do outro lado executamos mais uma exclusão do universal
 em $\forall x (\neg Q(x) \to R(x))$ e atribuímos novamente a mesma variável $t$. Com $\neg Q(t)$ e $\neg Q(t) \to R(t)$ 
 podemos realizar uma exclusão da implicaçã e por fim incluímos o universal em $\forall 
 I$.

\subsection{Quantificador existencial}
Lembrando a regra de exclusão do existencial, o que fazemos é que com $\exists x A(x)$, supomos um $y$
arbitrário que satisfaça $A(y)$, a partir desta premissa chegamos até $B$, uma fórmula que não contém
$y$ ou qualquer outra variável aberta em alguma hipótese não cancelada, e podemos concluir $B$.
\newline Já a regra da inclusão do existencial, se uma propriedade vale para um $y$ arbitrário, então
existe uma váriavel em que ela é válida.
\newline Iniciemos com uma demosntração simples de que se existe $x$ que satisfaça $A$ e $B$, então 
existe $x$ que satisfaça $A$.

\begin{center}
    \AxiomC{}
    \RightLabel{\scriptsize 1}
    \UnaryInfC{$\exists x (A(x) \land B(x))$}
    \AxiomC{}
    \RightLabel{\scriptsize 2}
    \UnaryInfC{$A(t) \land B(t)$}
    \UnaryInfC{$A(t)$}
    \UnaryInfC{$\exists x A(x)$}
    \RightLabel{\scriptsize 2}
    \BinaryInfC{$\exists x A(x)$}
    \RightLabel{\scriptsize 1}
    \UnaryInfC{$\exists x (A(x) \land B(x)) \to \exists x A(x)$}
    \DisplayProof
\end{center}
A primeira etapa foi a retirada da implicação, passando $\exists x (A(x) \land B(x))$ como uma
hipótese, e utilizando ela aplicamos a regra de exclusão do existencial, criando uma variável
$t$ arbitrária em que $A(x) \land B(x)$ sejá válido, com isso podemos concluir $A(t)$, note que em $2$
não podemos concluir $A(t)$, pois a variável ainda está aberta em $A(t)\land B(t)$, por isso inserimos
o existencial e concluímos $\exists x A(x)$, resultado que queríamos obter.
\newline O próximo exemplo relaciona os quantificadores universal e existencial, se para todo $x$ $A$
é válido, então existe algum $x$ que $A$ sejá válido.
\begin{center}
    \AxiomC{$\forall x A(x)$}
    \UnaryInfC{$A(t)$}
    \UnaryInfC{$\exists x A(x)$}
    \DisplayProof
\end{center}
Note que se $A(x)$ vale para todo $x$, também vale para um $t$ específico e no passo seguinte poderíamos
tanto utilizar a regra da inclusão do universal ou do existencial. Outro comentário relevante é que não
necessariamente precisávamos concluir o existencial utilizando a mesma variável $x$.
\newline Vamos provar mais uma relação entre os quantificadores, iremos que provar que se para todo $x$ 
não vale $A$, então não existe $x$ tal que $A$ valha:
\begin{center}
    \AxiomC{}
    \RightLabel{\scriptsize 2}
    \UnaryInfC{$\exists x A(x)$}
    \AxiomC{}
    \RightLabel{\scriptsize 3}
    \UnaryInfC{$A(t)$}
    \AxiomC{}
    \RightLabel{\scriptsize 1}
    \UnaryInfC{$\forall x \neg A(x)$}
    \UnaryInfC{$ \neg A(t)$}
    \BinaryInfC{$\bot $}
    \RightLabel{\scriptsize 3}
    \BinaryInfC{$\bot$}
    \RightLabel{\scriptsize 2}
    \UnaryInfC{$\neg \exists x A(x)$}
    \RightLabel{\scriptsize 1}
    \UnaryInfC{$\forall x \neg A(x) \to \neg \exists x A(x)$}
    \DisplayProof
\end{center}
Para a nossa dedução a primeira etapa foi desmontar a implicação, passando $\forall x \neg A(x)$ como
uma de nossas hipóteses, em seguida, como temos uma negação devemos chegar até ao falso. Utilizamos 
$\exists x A(x)$ como mais uma de nossas hipóteses e aplicamos a regra de exclusão do existencial,
dessa forma obtemos no mesmo ramo $A(t)$ e $\neg A(t)$, obtendo a contradição que estávamos procurando.
\subsection{Igualdade}
TO DO 
\subsection{Exercícios}
TO DO
\subsection{Dedução natural no LEAN}
No Lean a dedução ocorre de forma similar a dedução em primeira ordem, apenas devemos utilizar de novos símbolos e das 
regras de exclusão e inclusão dos quantificadores. Os símbolos $\exists $ e $\forall $ são escritos no Lean como \textbackslash exist 
e  \textbackslash all. Para a regra de exclusão do universal apenas passamos para uma proposição $\forall x A(x)$ uma letra,
por exemplo $t$, para termos $A(t)$. Para a inclusão do universal, devemos assumir uma letra, por exemplo $t$, utilizando o
"assume" e com esta letra livre de qualquer hipótese provamos $A(t)$, dessa forma o Lean é capaz de inferir $\forall x A(x)$.
\newline Vamos utilizando o Lean provar o nosso primeiro exemplo do quantificador universal, $\forall xP(x) \to \forall x Q(x) \to \forall x(P(x) \land Q(x))$:
\begin{lstlisting}
 variable U : Type
 variables P Q : Type $\to$ Prop

 example : ($\forall$x, P x) $\to$ ($\forall$x, Q x) $\to \forall$x P x $\land$ Q x :=  
 assume h$_1$ : $\forall$x, P x,
 assume h$_2$ : $\forall$x, Q x,
 assume t,
 have h$_3$ : P t, from h$_1$ t,
 have h$_4$ : Q t, from h$_2$ t,
 show P t $\land$ Q t, from and.intro h$_3$ h$_4$ 
\end{lstlisting}
Note que na linha 4 utilizamos da inclusão do universal, assumimos um $t$ e desejamos provar $P(t) \land Q(t)$ 
sem que $t$ possua qualquer restrição e nas linhas 5 e 6 utilizamos a exclusão do universal, temos fórmulas do
tipo $\forall x P(x)$ e passamos a letra $t$, obtendo $P(t)$. Além disso note nesse exemplo a importância da utilização
dos parenteses para delimitar a ação do quantificador, quando definimos o exemplo, caso não tivessemos colocado
o parenteses em  $(\forall x, P x)$, o primeiro quantificador $\forall x$ seria interpretado como se valesse para toda
a expressão, incluindo as duas implicações.
\newline Vamos também provar o segundo exemplo em Lean:
\begin{lstlisting}
variable U : Type
variables P Q R : U $\to$ Prop

example (h$_1$ : $\forall$ x, P x $\land \neg$ Q x) (h$_2$ : $\forall$ x, $\neg$ Q x $\to$ R x) :
$\forall$ x, R x :=
assume t,
have h$_3$ : P t $\land$ $\neg$ Q t, from h$_1$ t,
have h$_4$ : $\neg$ Q t $\to$ R t, from h$_2$ t,
show R t, from h$_4$ h$_3$.right
\end{lstlisting}
Note que neste simples exemplo, apesar de possuirmos fórmulas maiores, apenas realizamos simples regras de inclusão e exclusão
do universal.
\newline Para as regras do existencial, na exclusão utilizamos "exists.elim" seguida por uma proposição do tipo $\forall x A(x)$,
para provarmos $B$, devemos provar $ A y \to B$, ou seja, assumimos um $y$ e $A y$ e chegamos até $B$, concluindo assim a exclusão
do existencial. Para a inclusão do existencial utilizamos "exists.intro", devemos passar uma letra, por exemplo $t$, e uma prova
de que $A(t)$ vale.
\newline Vamos provar o nosso primeiro exemplo existencial com o Lean, $\exists x (A(x) \land B(x)) \to \exists x A(x)$:
\begin{lstlisting}
variable U : Type
variables A B : U $\to$ Prop

example : ($\exists$x, A x $\land$ B x) $\to \exists$x, A x :=
assume h$_1$ : $\exists$ x, A x $\land$ B x,
exists.elim h$_1$
    (assume t (h$_2$ : A t $\land$ B t)
        show $\exists$ x, A x, from exists.intro t h$_2$.left) 
\end{lstlisting}
\subsection{Exercícios}
\textbf{1.}Resolve as seguintes deduções naturais:
\newline \textbf{a)} $\forall x A(x) \to \neg \exists x \neg A(x)$.
\newline \textbf{SOLUÇÃO:}

\begin{lstlisting}
variable U : Type
variables A : U → Prop
    
    
--term mode
example : (∀ x, A x) → (¬ ∃ x, ¬ A x) :=
assume h1 : ∀ x, A x,
assume h2 : ∃ x, ¬ A x, show false, from
    (exists.elim h2
        (assume t (h3 : ¬ P t), 
            have h4 : A t, from h1 t, h3 h4))
    
--tatics
example : (∀ x, A x) → (¬ ∃ x, ¬ A x) :=
begin
intros h1,
intro h2,
apply exists.elim h2,
intro t,
intro h3,
exact h3 (h1 t)
end 
\end{lstlisting}
\textbf{b)} $\exists x \neg A (x) \to \neg \forall x A(x)$.
\newline \textbf{SOLUÇÃO:} 
\begin{lstlisting}
variable U : Type
variable A : U → Prop

--term mode
example : (∃ x, ¬ A x) → ¬ ∀ x, A x :=
assume h1 : ∃ x, ¬ A x,
assume h2 : ∀ x, A x, 
show false, from 
    (exists.elim h1
        (assume t (h3 : ¬ A t),
            have h4 : A t, from h2 t,
            h3 h4))
        
--tatics

example : (∃ x, ¬ A x) → ¬ ∀ x, A x :=
begin
intro h1,
intro h2,
apply exists.elim h1,
intro t,
intro h3,
exact h3 (h2 t)
end
\end{lstlisting}
\textbf{c)}
\textbf{d)}
\textbf{e)}
\newline \textbf{2.} Retornando ao capítulo anterior com o exercício de Ana, Cláudia e Maria e seus três vestidos
agora podemos resolver o problema utilizando de proposição de primeira ordem. Na situação, temos: 
\newline Três irmãs - Ana, Maria e Cláudia - foram a uma festa com vestidos de
cores diferentes. Uma vestia azul, a outra branco e a Terceira
preto. Chegando à festa, o anfitrião perguntou quem era cada uma
delas. As respostas foram:
\newline - A de azul respondeu: “Ana é a que está de branco”
\newline - A de branco falou: “Eu sou Maria”
\newline - A de preto disse:  “Cláudia é quem está de branco”
\newline O anfitrião foi capaz de identificar corretamente quem era cada pessoa
considerando que:
\newline - Ana sempre diz a verdade
\newline - Maria às vezes diz a verdade
\newline - Cláudia nunca diz a verdade
\newline Pensando um pouco sobre o problema, pode-se concluir que a Ana estava
com o vestido preto, a Cláudia com o branco e a Maria com o
azul.  
\newline \textbf{a)} Escreva as fórmulas de primeira ordem necessárias para o problema.
\newline \textbf{SOLUÇÃO:}
\begin{lstlisting}
section 
variable {pessoa : Type} 
variable {cor : Type}
variables {Ana Maria Claudia : pessoa}
variables {preto branco azul : cor}
variable {veste : pessoa → cor → Prop}

-- Restricoes:

-- Se uma usa uma cor, as outras nao podem usar essa cor
variable r1 : ∀c, ∀x, ∀y, (veste x c) ∧ ¬(x = y) → ¬(veste y c)
-- Se uma pessoa usa tal cor, nao pode usar as outras cores
variable r2 : ∀x, ∀c, ∀d, (veste x c) ∧ ¬(c = d) → ¬(veste x d)

-- Raciocinios basicos:

--Se Ana for a de azul, ela esta de branco
variable h1 : (veste Ana azul) → (veste Ana branco)
--Se Ana for a de branco, entao Maria esta de branco
variable h2 : (veste Ana branco) → (veste Maria branco)
--Se ana for de preto, entao Claudia esta de branco
variable h3 : (veste Ana preto) → (veste Claudia branco)
--Se Ana não usa azul e nem branco, ela esta de preto
variable h4 : ¬ (veste Ana azul) ∧ ¬ (veste Ana branco)  →  (veste Ana preto)
--Se Ana veste preto e Claudia veste branco, entao Maria esta de azul
variable h5 : (veste Ana preto) ∧ (veste Claudia branco) →  (veste Maria azul)

variable d1 : ¬ (azul = branco)
variable d2 : ¬ (Ana = Maria)
-- prova de que Ana veste preto
lemma Ana_p : (veste Ana preto) :=
have n1 : ¬ (veste Ana azul), from 
assume m1 : veste Ana azul, show false, from 
    have q1 : veste Ana branco, from h1 m1,
    have q2 : ¬ veste Ana branco, from (r2 Ana azul branco (
        and.intro m1 d1)),
q2 q1,
have n2 : ¬ (veste Ana branco), from
assume m2 : veste Ana branco, show false, from
    have q1 : veste Maria branco, from h2 m2,
    have q2 : ¬ veste Maria branco, from (r1 branco Ana Maria (and.intro m2 d2)),
q2 q1,
show veste Ana preto, from h4 (and.intro n1 n2)

theorem solution : (veste Ana preto) ∧ (veste Claudia branco) ∧ (veste Maria azul) :=
have n1 : veste Ana preto, from (Ana_p r1 r2 h1 h2 h4 d1 d2),
have n2 : veste Claudia branco, from h3 (Ana_p r1 r2 h1 h2 h4 d1 d2),
have n3 : veste Maria azul, from h5 (and.intro n1 n2),
and.intro n1 (and.intro n2 n3)

end
\end{lstlisting}
\textbf{SOLUÇÃO COM TÁTICAS}:
\begin{lstlisting}
    
\end{lstlisting}