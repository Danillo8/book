\section{Dedução Natural}
Como visto na seção anterior, devemos estabelecer as regras de dedução para os quantificadores e para a igualdade. Listamos elas inicialmente:
\newline \textbf{Quantificador universal:}
 \newline 
 \begin{center}
    \AxiomC{$A(y)$}
    \RightLabel{\scriptsize $\forall I$}
    \UnaryInfC{$\forall x A(x)$}
    \DisplayProof
\end{center}
    Para inserir o quantificador universal devemos ter que a variável $y$ não deve estar atrelada em nenhuma hipótese, isto é, em todas as hipóteses não canceladas ela não pode ser livre. 
\begin{center}

   
    \AxiomC{$\forall x A(x)$}
    \RightLabel{\scriptsize $\forall E$}
    \UnaryInfC{$A(t)$}
    \DisplayProof   
 \end{center}


 \textbf{Quantificador existencial:}
 \begin{center}
     \AxiomC{$A(t)$}
     \RightLabel{\scriptsize $\exists I$}
     \UnaryInfC{$\exists x A(x)$}
     \DisplayProof

     \AxiomC{$\exists x A(x)$}
        \AxiomC{}
        \UnaryInfC{$A(t)$}
        \alwaysNoLine
        \UnaryInfC{$\vdots$}
        \UnaryInfC{$B$}
        \alwaysSingleLine
        \RightLabel{\scriptsize $\exists E$}
        \BinaryInfC{$B$}
     \DisplayProof
\end{center}

Para retirarmmos o quantificador existencial, a variável $t$ não pode estar livre em $B$, isto é, ela não deve estar livre em qualquer hipótese não cancelada.

\textbf{Igualdade:}
\begin{center}
    \AxiomC{}
    \RightLabel{\scriptsize refl}
    \UnaryInfC{$t=t$}
    \DisplayProof

    \AxiomC{$t=s$}
    \RightLabel{\scriptsize sim}
    \UnaryInfC{$s=t$}
    \DisplayProof

    \AxiomC{$t=s$}
    \AxiomC{$s=v$}
    \RightLabel{\scriptsize trans}
    \BinaryInfC{$t=v$}
    \DisplayProof

    \AxiomC{$t=s$}
    \RightLabel{\scriptsize subs}
    \UnaryInfC{$r(t) = r(s)$}
    \DisplayProof

    \AxiomC{$t=s$}
    \AxiomC{$P(t)$}
    \RightLabel{\scriptsize subs}
    \BinaryInfC{$P(s)$}
    \DisplayProof
\end{center}
Iremos passar por cada um destas regras para apresentá-las com o auxílio de exemplos.

\subsection{Quantificador universal}
Como primeiro exemplo segue abaixo uma dedução natural de $(\forall x P(x) \to \forall x Q(x)) \to \forall x (P ( x) \to Q (x))$. Note que apesar de parecer uma implicação de duas fórmulas idênticas, na premissa a propriedade $P$ recebe uma variável e a propriedade $Q$ pode receber outra variável, enquanto que na conclusão o valor que $x$ assume é o mesmo para $P$ e $Q$.

\begin{center}
    \AxiomC{}
    \RightLabel{\scriptsize 1}
    \UnaryInfC{$\forall x P(x) \to \forall x Q(x)$}
    \RightLabel{\scriptsize $\forall E$}
    \UnaryInfC{$P(t) \to \forall x Q(x)$}
    \RightLabel{\scriptsize $\forall E$}
    \UnaryInfC{$P(t) \to Q(t)$}
    \RightLabel{\scriptsize $\forall I$}
    \UnaryInfC{$\forall x (P(x) \to Q(x))$}
    \RightLabel{\scriptsize 1}
    \UnaryInfC{$\forall x P(x) \to \forall x Q(x) \to \forall x (P(x) \to Q(x))$}
    \DisplayProof
\end{center}

O primeiro passo da demonstração será a exclusão da implicação (a implicação principal da fórmula), assumindo $\forall x P(x) \to \forall x Q(x)$ como nossa hipótese. Aplicamos uma primeira exclusão do universal em $\forall x P(x)$ e novamente aplicamos a exclusão do universal em $\forall x Q(x)$. Note que ao definirmos a variável com o mesmo nome $t$ em ambas as exclusões do universal foi o que permitiu inserir um universal que englobe ambas variáveis, que é o passa seguinte.
\newline Vamos para um outro exemplo, utilizando as hipóteses $\forall x (\neg Q(x) \to R(x))$ e $\forall x(P(x) \land \neg Q(x))$ provaremos $\forall x R(x)$:

\begin{center}
    \AxiomC{$\forall x(P(x) \land \neg Q(x))$}
    \RightLabel{\scriptsize $\forall E_1$}
    \UnaryInfC{$P(t) \land \neg Q(t)$}
    \RightLabel{\scriptsize $\land E_2$}
    \UnaryInfC{$\neg Q(t)$}
    \AxiomC{$\forall x (\neg Q(x) \to R(x))$}
    \RightLabel{\scriptsize $\forall E_3$}
    \UnaryInfC{$\neg Q(t) \to R(t)$}
    \RightLabel{\scriptsize $\to E_4$}
    \BinaryInfC{$R(t)$}
    \RightLabel{\scriptsize $\forall I_5$}
    \UnaryInfC{$\forall x R(x)$}
    \DisplayProof
\end{center}
Vamos inicialmente utilizar de nossas hipóteses, em $\forall E_1$ nós executamos a primeira exclusão do universal, utilizando a variável $t$ e em seguida executamos a exclusão do $\land$ visto que não é necessário o $P(t)$ na dedução. Do outro lado executamos mais uma exclusão do universal em $\forall E_3$ e atribuímos novamente a mesma variável $t$. Com $\neg Q(t)$ e $\neg Q(t) \to R(t)$ podemos realizar uma exclusão da implicação em $\to E_4$ e por fim incluímos o universal em $\forall I_5$.

\subsection{Quantificador existencial}
Utilizando as propriedades $homem$, $mortal$, $socrates$ e as premissas $\forall x (homem(x) \to mortal(x)$ e $\exists x(socrates(x) \land homem(x))$ para provar $\exists x socrates(x) \land mortal(x)$.
\begin{center}
    \AxiomC{$\forall x (homem (x) \to mortal(x))$}
    \UnaryInfC{$homem(t) \to mortal(t)$}
    \AxiomC{$\exists x (socrates(x) \land homem(x))$}
    \UnaryInfC{$ socrates(t) \land homem(t)$}
    \UnaryInfC{$homem(t)$}
    \BinaryInfC{$mortal(t)$}
    \AxiomC{$\exists x (socrates(x) \land homem(x))$}
    \UnaryInfC{$socrates(t) \land homem(t)$}
    \UnaryInfC{$socrates(t)$}
    \BinaryInfC{$mortal(t) \land socrates(t)$}
    \UnaryInfC{$\exists x (mortal(x) \land socrates(x))$}
    \DisplayProof
\end{center}
