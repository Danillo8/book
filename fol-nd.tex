\section{Dedução Natural}
Como visto na seção anterior devemos estabelecer as regras de inclusão e exclusão
de dedução para os quantificadores e para a igualdade. Listamos elas inicialmente:
\newline \textbf{Quantificador universal:}
 \newline 
 \begin{center}
    \AxiomC{$A(y)$}
    \RightLabel{\scriptsize $\forall I$}
    \UnaryInfC{$\forall x A(x)$}
    \DisplayProof
\end{center}
    Para inserir o quantificador universal devemos ter 
    que a variável $y$ não deve estar atrelada em nenhuma
     hipótese, isto é, em todas as hipóteses não canceladas
      ela não pode ser livre. 
\begin{center}

   
    \AxiomC{$\forall x A(x)$}
    \RightLabel{\scriptsize $\forall E$}
    \UnaryInfC{$A(t)$}
    \DisplayProof   
 \end{center}


 \textbf{Quantificador existencial:}
 \begin{center}
     \AxiomC{$A(t)$}
     \RightLabel{\scriptsize $\exists I$}
     \UnaryInfC{$\exists x A(x)$}
     \DisplayProof

     \AxiomC{$\exists x A(x)$}
        \AxiomC{}
        \UnaryInfC{$A(t)$}
        \alwaysNoLine
        \UnaryInfC{$\vdots$}
        \UnaryInfC{$B$}
        \alwaysSingleLine
        \RightLabel{\scriptsize $\exists E$}
        \BinaryInfC{$B$}
     \DisplayProof
\end{center}

Para retirarmmos o quantificador existencial, a variável $t$ não pode estar livre em $B$, isto é, ela não 
deve estar livre em qualquer hipótese não cancelada.

\textbf{Igualdade:}
\begin{center}
    \AxiomC{}
    \RightLabel{\scriptsize refl}
    \UnaryInfC{$t=t$}
    \DisplayProof

    \AxiomC{$t=s$}
    \RightLabel{\scriptsize sim}
    \UnaryInfC{$s=t$}
    \DisplayProof

    \AxiomC{$t=s$}
    \AxiomC{$s=v$}
    \RightLabel{\scriptsize trans}
    \BinaryInfC{$t=v$}
    \DisplayProof

    \AxiomC{$t=s$}
    \RightLabel{\scriptsize subs}
    \UnaryInfC{$r(t) = r(s)$}
    \DisplayProof

    \AxiomC{$t=s$}
    \AxiomC{$P(t)$}
    \RightLabel{\scriptsize subs}
    \BinaryInfC{$P(s)$}
    \DisplayProof
\end{center}
Iremos passar por cada um destas regras para apresentá-las com o auxílio de exemplos.

\subsection{Quantificador universal}
Como primeiro exemplo segue abaixo uma dedução natural de 
$(\forall x P(x) \to \forall x Q(x)) \to \forall x (P ( x) \to Q (x))$. 
Note que apesar de parecer uma implicação de duas fórmulas idênticas, na premissa a propriedade
 $P$ recebe uma variável e a propriedade $Q$ pode receber outra variável, enquanto que na conclusão
  o valor que $x$ assume é o mesmo para $P$ e $Q$.

\begin{center}
    \AxiomC{}
    \RightLabel{\scriptsize 1}
    \UnaryInfC{$\forall x P(x) \to \forall x Q(x)$}
    \RightLabel{\scriptsize $\forall E$}
    \UnaryInfC{$P(t) \to \forall x Q(x)$}
    \RightLabel{\scriptsize $\forall E$}
    \UnaryInfC{$P(t) \to Q(t)$}
    \RightLabel{\scriptsize $\forall I$}
    \UnaryInfC{$\forall x (P(x) \to Q(x))$}
    \RightLabel{\scriptsize 1}
    \UnaryInfC{$\forall x P(x) \to \forall x Q(x) \to \forall x (P(x) \to Q(x))$}
    \DisplayProof
\end{center}

O primeiro passo será a exclusão da implicação (a implicação principal da fórmula), 
assumindo $\forall x P(x) \to \forall x Q(x)$ como nossa hipótese. Aplicamos uma primeira exclusão do
 universal em $\forall x P(x)$ e novamente aplicamos a exclusão do universal em $\forall x Q(x)$. Note que 
 ao definirmos a variável com o mesmo nome $t$ em ambas as exclusões do universal foi o que permitiu inserir 
 um universal que englobe ambas variáveis, que é o passa seguinte.
\newline Vamos para um outro exemplo, utilizando as hipóteses 
$\forall x (\neg Q(x) \to R(x))$ e $\forall x(P(x) \land \neg Q(x))$ provaremos $\forall x R(x)$:

\begin{center}
    \AxiomC{$\forall x(P(x) \land \neg Q(x))$}
    \RightLabel{\scriptsize $\forall E_1$}
    \UnaryInfC{$P(t) \land \neg Q(t)$}
    \RightLabel{\scriptsize $\land E_2$}
    \UnaryInfC{$\neg Q(t)$}
    \AxiomC{$\forall x (\neg Q(x) \to R(x))$}
    \RightLabel{\scriptsize $\forall E_3$}
    \UnaryInfC{$\neg Q(t) \to R(t)$}
    \RightLabel{\scriptsize $\to E_4$}
    \BinaryInfC{$R(t)$}
    \RightLabel{\scriptsize $\forall I_5$}
    \UnaryInfC{$\forall x R(x)$}
    \DisplayProof
\end{center}
Vamos inicialmente utilizar de nossas hipóteses, em $\forall E_1$ nós executamos a primeira 
exclusão do universal, utilizando a variável $t$ e em seguida executamos a exclusão do $\land$ 
visto que não é necessário o $P(t)$ na dedução. Do outro lado executamos mais uma exclusão do universal
 em $\forall E_3$ e atribuímos novamente a mesma variável $t$. Com $\neg Q(t)$ e $\neg Q(t) \to R(t)$ 
 podemos realizar uma exclusão da implicação em $\to E_4$ e por fim incluímos o universal em $\forall 
 I_5$.

\subsection{Quantificador existencial}
Lembrando a regra de exclusão do existencial, o que fazemos é que com $\exists x A(x)$, supomos um $y$
arbitrário que satisfaça $A(y)$, a partir desta premissa chegamos até $B$, uma fórmula que não contém
$y$ ou qualquer outra variável aberta em alguma hipótese não cancelada, e podemos concluir $B$.
\newline Já a regra da inclusão do existencial, se uma propriedade vale para um $y$ arbitrário, então
existe uma váriavel em que ela é válida.
\newline Iniciemos com uma demosntração simples de que se existe $x$ que satisfaça $A$ e $B$, então 
existe $x$ que satisfaça $A$.

\begin{center}
    \AxiomC{}
    \RightLabel{\scriptsize 1}
    \UnaryInfC{$\exists x (A(x) \land B(x))$}
    \AxiomC{}
    \RightLabel{\scriptsize 2}
    \UnaryInfC{$A(t) \land B(t)$}
    \UnaryInfC{$A(t)$}
    \UnaryInfC{$\exists x A(x)$}
    \RightLabel{\scriptsize 2}
    \BinaryInfC{$\exists x A(x)$}
    \RightLabel{\scriptsize 1}
    \UnaryInfC{$\exists x (A(x) \land B(x)) \to \exists x A(x)$}
    \DisplayProof
\end{center}
A primeira etapa foi a retirada da implicação, passando $\exists x (A(x) \land B(x))$ como uma
hipótese, e utilizando ela aplicamos a regra de exclusão do existencial, criando uma variável
$t$ arbitrária em que $A(x) \land B(x)$ sejá válido, com isso podemos concluir $A(t)$, note que em $2$
não podemos concluir $A(t)$, pois a variável ainda está aberta em $A(t)\land B(t)$, por isso inserimos
o existencial e concluímos $\exists x A(x)$, resultado que queríamos obter.
\newline O próximo exemplo relaciona os quantificadores universal e existencial, se para todo $x$ $A$
é válido, então existe algum $x$ que $A$ sejá válido.
\begin{center}
    \AxiomC{$\forall x A(x)$}
    \UnaryInfC{$A(t)$}
    \UnaryInfC{$\exists x A(x)$}
    \DisplayProof
\end{center}
Note que se $A(x)$ vale para todo $x$, também vale para um $t$ específico e no passo seguinte poderíamos
tanto utilizar a regra da inclusão do universal ou do existencial. Outro comentário relevante é que não
necessariamente precisávamos concluir o existencial utilizando a mesma variável $x$.
\newline Vamos provar mais uma relação entre os quantificadores, iremos que provar que se para todo $x$ 
não vale $A$, então não existe $x$ tal que $A$ valha:
\begin{center}
    \AxiomC{}
    \RightLabel{\scriptsize 2}
    \UnaryInfC{$\exists x A(x)$}
    \AxiomC{}
    \RightLabel{\scriptsize 3}
    \UnaryInfC{$A(t)$}
    \AxiomC{}
    \RightLabel{\scriptsize 1}
    \UnaryInfC{$\forall x \neg A(x)$}
    \UnaryInfC{$ \neg A(t)$}
    \BinaryInfC{$\bot $}
    \RightLabel{\scriptsize 3}
    \BinaryInfC{$\bot$}
    \RightLabel{\scriptsize 2}
    \UnaryInfC{$\neg \exists x A(x)$}
    \RightLabel{\scriptsize 1}
    \UnaryInfC{$\forall x \neg A(x) \to \neg \exists x A(x)$}
    \DisplayProof
\end{center}
Para a nossa dedução a primeira etapa foi desmontar a implicação, passando $\forall x \neg A(x)$ como
uma de nossas hipóteses, em seguida, como temos uma negação devemos chegar até ao falso. Utilizamos 
$\exists x A(x)$ como mais uma de nossas hipóteses e aplicamos a regra de exclusão do existencial,
dessa forma obtemos no mesmo ramo $A(t)$ e $\neg A(t)$, obtendo a contradição que estávamos procurando.
\subsection{Igualdade}
TO DO 
\subsection{Exercícios}
TO DO
\subsection{Dedução natural no LEAN}
No Lean a dedução ocorre de forma similar a dedução em primeira ordem, apenas devemos utilizar de novos símbolos e das 
regras de exclusão e inclusão dos quantificadores. Os símbolos $\exists $ e $\forall $ são escritos no Lean como \textbackslash exist 
e  \textbackslash all. Para a regra de exclusão do universal apenas passamos para uma proposição $\forall x A(x)$ uma letra,
por exemplo $t$, para termos $A(t)$. Para a inclusão do universal, devemos assumir uma letra, por exemplo $t$, utilizando o
"assume" e com esta letra livre de qualquer hipótese provamos $A(t)$, dessa forma o Lean é capaz de inferir $\forall x A(x)$.
\newline Vamos utilizando o Lean provar o nosso primeiro exemplo do quantificador universal, $\forall xP(x) \to \forall x Q(x) \to \forall x(P(x) \land Q(x))$:
\begin{lstlisting}
 example : $\forall$x P(x) $\to$ $\forall$x Q(x) $\to \forall$x (P(x) $\land$ Q (x)) :=  
 assume h$_1$ : $\forall$x P(x),
 assume h$_2$ : $\forall$x Q(x),
 assume t,
 have h$_3$ : P(t), from h$_1$ t,
 have h$_4$ : Q(t), from h$_2$ t,
 show P(t) $\land$ Q(t), from and.intro h$_3$ h$_4$ 
\end{lstlisting}
Note que na linha 4 utilizamos da inclusão do universal, assumimos um $t$ e desejamos provar $P(t) \land Q(t)$ 
sem que $t$ possua qualquer restrição e nas linhas 5 e 6 utilizamos a exclusão do universal, temos fórmulas do
tipo $\forall x P(x)$ e passamos a letra $t$, obtendo $P(t)$.
\newline Para as regras do existencial, na exclusão utilizamos "exists.elim" seguida por uma proposição do tipo $\forall x A(x)$,
para provarmos $B$, devemos provar $ A y \to B$, ou seja, assumimos um $y$ e $A y$ e chegamos até $B$, concluindo assim a exclusão
do existencial. Para a inclusão do existencial utilizamos "exists.intro", devemos passar uma letra, por exemplo $t$, e uma prova
de que $A(t)$ vale.