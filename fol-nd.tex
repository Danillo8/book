\section{Dedução Natural}
Como visto na seção anterior, devemos estabelecer as regras de dedução para os quantificadores e para a igualdade. Listamos elas inicialmente:
\newline \textbf{Quantificador universal:}
 \newline 
 \begin{center}
    \AxiomC{$A(y)$}
    \RightLabel{\scriptsize $\forall I$}
    \UnaryInfC{$\forall x A(x)$}
    \DisplayProof
   
    \AxiomC{$\forall x A(x)$}
    \RightLabel{\scriptsize $\forall E$}
    \UnaryInfC{$A(t)$}
    \DisplayProof   
 \end{center}


 \textbf{Quantificador existencial:}
 \begin{center}
     \AxiomC{$A(t)$}
     \RightLabel{\scriptsize $\exists I$}
     \UnaryInfC{$\exists x A(x)$}
     \DisplayProof

     \AxiomC{$\exists x A(x)$}
        \AxiomC{}
        \UnaryInfC{$A(t)$}
        \alwaysNoLine
        \UnaryInfC{$\vdots$}
        \UnaryInfC{$B$}
        \alwaysSingleLine
        \RightLabel{\scriptsize $\exists E$}
        \BinaryInfC{$B$}
     \DisplayProof
\end{center}

\textbf{Igualdade:}
\begin{center}
    \AxiomC{}
    \RightLabel{\scriptsize refl}
    \UnaryInfC{$t=t$}
    \DisplayProof

    \AxiomC{$t=s$}
    \RightLabel{\scriptsize sim}
    \UnaryInfC{$s=t$}
    \DisplayProof

    \AxiomC{$t=s$}
    \AxiomC{$s=v$}
    \RightLabel{\scriptsize trans}
    \BinaryInfC{$t=v$}
    \DisplayProof

    \AxiomC{$t=s$}
    \RightLabel{\scriptsize subs}
    \UnaryInfC{$r(t) = r(s)$}
    \DisplayProof

    \AxiomC{$t=s$}
    \AxiomC{$P(t)$}
    \RightLabel{\scriptsize subs}
    \BinaryInfC{$P(s)$}
    \DisplayProof
\end{center}