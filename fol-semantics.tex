\section{Semântica}
    
    A seção anterior abordou a sintaxe da lógica de primeira ordem, ou seja, apresentou os símbolos e estruturas das fórmulas deste sistema. A semântica, por outro lado, define o valor-verdade das fórmulas pela \textbf{interpretação} e \textbf{valoração}.
   
   Analisemos alguns exemplos deste sistema lógico.
   Sejam:
   \begin{itemize}
       \item $\mathbb{N}$ um domínio escolhido;
       \item $0, 1, 2, 3$ símbolos constantes;
       \item \textit{adicao} e \textit{sucessor} os símbolos de função que operam neste domínio;
       \item  \textit{par}, \textit{impar}, \textit{menorDoQue},\textit{menorIgualDoQue} os simbolos de predicado .
   \end{itemize}
    A expressão 
   \begin{center}
       $\forall x \ ( \ menorDoQue(0, x) \ )$
   \end{center}
   
   é falsa, uma vez que para $x$ assumindo o valor de $0$, a relação \textit{menorDoQue} não é válida. Já a expressão
   
   \begin{center}
       $\forall x \ ( \ menorIgualDoQue( 0, x) \ )$
   \end{center}
    
    é verdadeira no domínio $\mathbb{N}$, porém se o domínio de interesse fosse $\mathbb{Z}$ a expressão se tornaria falsa.
    
    Logo, o valor verdade de uma sentença depende de como os quantificadores, o domínio, funções, predicados e relações são interpretados. 
    Porém, há algumas fórmulas que assumem sempre valor verdadeiro independente de qual for a interpretação, análogo à tautologia da lógica proposicional:
    
    \begin{center}
        $\forall x \ ( \  par(x) \rightarrow par(x) \ )$
    \end{center}
    
    Sentenças assim são chamadas \textbf{válidas}.
    
    Analogamente à valoração em lógica proposicional, há o \textbf{modelo} em lógica de primeira ordem. Enquanto a valoração permitia que atribuíssemos valores-verdade ( V ou F ) à todas as fórmulas da lógica proposicional, a escolha de um modelo permite a atribuição de valores-verdade a todas as sentenças da lógica de primeira ordem.
    
    \subsection{Interpretações}
    
    Usamos anteriormente alguns símbolos para representar predicados e constantes. Alguns deles:
    
    \begin{center}
        $0$, $1$, $ par$, $maiorDoQue$
    \end{center}
    
    
    Estes símbolos são autodescritivos considerando o domínio $\mathbb{N}$, e se torna natural sua valoração. Agora, sejam os seguintes predicados no mesmo domínio:
    
    \begin{center}
        $ligeiro$, $contente$, $fácil$
    \end{center}
    
    Quando o predicado $ligeiro$ é verdadeiro? \\
    Não conseguimos responder, uma vez que não foram dadas informações suficientes.
    Se $ligeiro$ são os números ímpares, então $2, 4, 6, ...$ são  $ligeiro$ e   $1, 3, 5, ...$ não são. Seja $contente$ os múltiplos de $3$. Então $6$ é $ligeiro$ e $contente$. Se $ligeiro$ fossem os números da sequência de Fibonacci, então $6$ seria $contente$, mas não seria $ligeiro$.
    Cada explicação dada ( ímpar, múltiplos de $3$, números da sequência de Fibonacci) são $interpretações$. E vemos que é necessário a interpretação para a valoração. 
    
    
    Assim como os predicados, podemos interpretar as funções, relações e constantes:
    
    \begin{itemize}
        \item A interpretação de um predicado unário $P$ é um conjunto de elementos do domínio os quais $P$ é verdadeiro. 
        \item Para uma relação $R$ com aridade $n$, a interpretação é o conjunto de todas as tuplas com $n$ elementos para as quais $R$ é verdadeiro.
        \item E por fim, a interpretaçãode de uma função $f$ com aridade $n$, é uma função que relaciona $n$ elementos do domínio a outro elemento também do domínio.\\
     \end{itemize}
     
     É importante ressaltar a diferença entre símbolo sintático e semântica do predicado, função, relação e constante. Veja que não faz sentido escrevermos a relação $sucessorDe(4,3)$, pois $sucessorDe$ é um símbolo sintático sem significado por si só. \\
     Outra distinção importante é entre os objetos dos domínio e os símbolos constantes. Se considerarmos o domínio U de todas as cores, conhecemos os objetos deste domínio, mas podemos escrever o símbolo constante $verde$ e podemos interpretá-lo com verde ou como rosa, objetos do domínio.
     De maneira análoga, podemos definir os símbolos $0$, $1$, $2$ e interpretálos como os objetos do domínio $0$, $1$ e $2$.
     
    \subsection{Verdade e modelos}
       
    \subsection{Exemplos}
    
    \subsection{Validação e consequência lógica}
    
    \subsection{Correção e completude}
    
