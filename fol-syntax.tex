\section{Sintaxe}

    A sintaxe de um sistema lógico aborda basicamente os símbolos que são utilizados para representá-lo. Portanto, nesta seção, serão abordados funções, predicados, relações e quantificadores, dentre eles $\forall$ e $\exists$.

    \subsection{Funções, Predicados e Relações}

        Dentro de Lógica de Primeira Ordem, funções, predicados e relações são mapeamentos que, dado algum elemento do domínio, retornam uma proposição ou outro elemento do domínio. Parece confuso? Vamos olhar um exemplo.

        \begin{itemize}
            \item $x$ natural é par ou ímpar.
        \end{itemize}

        Nosso domínio, neste caso, são os números naturais ($\mathbb{N}$). Dizemos que \textit{par} é "algo" que recebe um número e retorna $V$ ou $F$. Chamamos isso de \textbf{predicado}. Sendo assim, podemos escrever:

        \begin{itemize}
            \item $par(x) \lor impar(x)$.
        \end{itemize}

        A partir deste exemplo, podemos extrair alguns símbolos para exemplificar a construção de um sistema lógico de primeira ordem.

        \begin{itemize}
            \item O domínio são os naturais;
            \item Os objetos são os números $0$, $1$, $2$ etc.;
            \item Existem \textbf{funções}, como \textit{adição} e \textit{subtração}, que recebem (zero ou mais) números e retornam outros números;
            \item Existem \textbf{predicados}, como \textit{par} e \textit{ímpar}, que recebem um número e retornam $V$ ou $F$;
            \item Existem \textbf{relações}, como \textit{igual} e \textit{menor}, que recebem dois números e retornam $V$ ou $F$.
        \end{itemize}

        Os objetos pertencentes ao domínio, chamados constantes, como o $1$ e o $4x$, no caso do domínio dos naturais, podem ser considerados funções que tomam zero elementos. Além disso, podemos considerar predicados que tomam zero elementos como os valores lógicos $\top$ e $\bot$.

        Em geral, elementos do domínio (incluindo funções de elementos) são chamados de \textbf{termos}. Alguns exemplos:

        \begin{itemize}
            \item $5$
            \item $sucessor(10)$ (função sucessora, retorna o elemento acrescido de $1$)
            \item $33+44$
        \end{itemize}

        Expressões do tipo $V$ ou $F$ são chamadas \textbf{fórmulas}:

        \begin{itemize}
            \item $maiorDoQue(1,2)$ ($1 > 2$)
            \item $par(10) \lor impar(5)$
            \item $2=2$
        \end{itemize}

        