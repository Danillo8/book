\chapter{Lógica de Primeira Ordem}
    
    Até agora, vimos Lógica Proposicional, em que variáveis proposicionais podem ser valoradas como verdadeiras ou falsas. Esse sistema lógico, porém, contém limitações. Considere, por exemplo, a afirmação de que todo natural é maior do que ou igual a zero.
    Ou a afirmação de que existe número natural que é primo e é par. Como representar isso em Lógica Propocional? Precisamos de um sistema lógico que saiba lidar com objetos e suas propriedades e relações.

    Talvez esses exemplos não tenham demonstrado a necessidade de tal sistema lógico. Porém, acredite: ao final deste capítulo, essa necessidade ficará evidente.

    \section{Sintaxe}

    A sintaxe de um sistema lógico aborda basicamente os símbolos que são utilizados para representá-lo. Portanto, nesta seção, serão abordados funções, predicados, relações e quantificadores, dentre eles $\forall$ e $\exists$.

    \subsection{Funções, Predicados e Relações}

        Dentro de Lógica de Primeira Ordem, funções, predicados e relações são mapeamentos que, dado algum elemento do domínio, retornam uma proposição ou outro elemento do domínio. Parece confuso? Vamos olhar um exemplo.

        \begin{center}
            $x$ natural é par ou ímpar.            
        \end{center}

        Nosso domínio, neste caso, são os números naturais ($\mathbb{N}$). Dizemos que \textit{par} é ``algo"\space que recebe um número e retorna $V$ ou $F$. Chamamos isso de \textbf{predicado}. Sendo assim, podemos escrever:

        \begin{center}
            $par(x) \lor impar(x)$.
        \end{center}

        A partir deste exemplo, podemos extrair alguns símbolos para exemplificar a construção de um sistema lógico de primeira ordem.

        \begin{itemize}
            \item O domínio são os naturais;
            \item Os objetos são os números $0$, $1$, $2$ etc.;
            \item Existem \textbf{funções}, como \textit{adição} e \textit{subtração}, que recebem (zero ou mais) números e retornam outros números;
            \item Existem \textbf{predicados}, como \textit{par} e \textit{ímpar}, que recebem um número e retornam $V$ ou $F$;
            \item Existem \textbf{relações}, como \textit{igual} e \textit{menor}, que recebem dois números e retornam $V$ ou $F$.
        \end{itemize}

        Os objetos pertencentes ao domínio, chamados constantes, como o $1$ e o $4$, no exemplo, podem ser considerados funções que tomam zero elementos. Além disso, podemos considerar predicados que tomam zero elementos como os valores lógicos $\top$ e $\bot$.

        Expressões que representam elementos do domínio(incluindo funções de elementos) são chamados de \textbf{termos}. Alguns exemplos:

        \begin{itemize}
            \item $5$
            \item $sucessor(10)$ (função sucessora, retorna o elemento acrescido de $1$)
            \item $33+44$
        \end{itemize}

        Observe que o símbolo para a função de adição está ``infixo"; poderíamos ter representado como $+(33, 44)$ ou $adicao(33, 44)$.
        
        Expressões que retornam $V$ ou $F$ são chamadas \textbf{fórmulas}:

        \begin{itemize}
            \item $maiorDoQue(1,2)$ ($1 > 2$)
            \item $par(10) \lor impar(5)$
            \item $2=2$
        \end{itemize}

        O Lean é muito eficiente em expressar Lógica de Primeira Ordem. Vejamos nosso exemplo:

        \begin{lstlisting}
constant U : Type
constant zero : U
constant par : U → Prop
constant primo : U → Prop
constant igual : U → U → Prop
constant adicao : U → U → U
\end{lstlisting}

        Pelo fato de que o Lean é baseado em \textit{Teoria dos Tipos}, declaramos um novo tipo \lstinline{U}. Podemos intuitivamente considerá-lo como ``universo"\space ou ``domínio". Por exemplo, o conjunto dos naturais.

        Foi declarado um objeto chamado \lstinline{zero} do tipo \lstinline{U} (nossa analogia com o zero natural).
        \lstinline{par} é um predicado, pois toma um elemento do tipo \lstinline{U} e retona um elemento do tipo proposição (\lstinline{Prop}).

        \lstinline{adicao} é uma função que toma dois elementos do tipo \lstinline{U} e retorna outro do mesmo tipo. Ora, podemos constatar:

        \begin{lstlisting}
#check par zero
#check adicao zero zero
#check par (adicao zero zero)
\end{lstlisting}

        O \lstinline{#check} da linha $1$ informa que a expressão tem tipo \lstinline{Prop}; na linha $2$, tipo \lstinline{U}; e, na linha $3$, tipo \lstinline{Prop}. Importante observar o papel dos parênteses acima, para que \lstinline{par} receba apenas um elemento.

        Uma função (ou relação) que recebe mais de um elemento tem notação \lstinline{U → U → U}. A notação para predicados (\lstinline{U → Prop}) e relações (\lstinline{U → U → Prop}) funciona como se ambas fossem funções, porém retornassem \lstinline{Prop}. 

        Vários conjuntos estão nas bibliotecas padrão do Lean, como os naturais, utilizados nos exemplos anteriores. O comando para o símbolo \lstinline{ℕ} é \lstinline{\nat} ou \lstinline{\N}.

        \begin{lstlisting}
constant zero : ℕ
#check zero + zero
\end{lstlisting}

        O \lstinline{#check} da linha $2$ retorna algo do tipo \lstinline{ℕ}.

        Podemos misturar Lógica Proposicional com Lógica de Primeira Ordem:

        \begin{lstlisting}
constant U : Type
constant zero : U
constant par : U → Prop
constant primo : U → Prop
constant igual : U → U → Prop
constant adicao : U → U → U

#check ¬ (par zero ∨ par (adicao zero zero)) ∧ primo zero
\end{lstlisting}

        E o \lstinline{#check} nos retorna algo do tipo \lstinline{Prop}.

    \subsection{Quantificador Universal}

        Grande parte do poder da Lógica Proposicional se deve aos quantificadores.
        O símbolo $\forall$ é o quantificador universal, que representa ``para todo".
        Quando ele é seguido de uma variável e de uma expressão, ele indica que aquela expressão é verdadeira para toda variável do domínio.
        Por exemplo:

        \begin{itemize}
            \item $\forall x \ (par(x) \lor impar(x))$
            \item $\forall y \ (par(y) \rightarrow impar(y + 1))$
        \end{itemize}

        A primeira expressão nos diz que todo número é par ou ímpar (no caso dos naturais).
        A segunda diz que, para todo número, o fato dele ser par implica que seu sucessor é ímpar.
        
        No Lean, é possível obter o símbolo \lstinline{∀} digitando \lstinline{\all}. Os exemplos anteriores ficam assim:

        \begin{lstlisting}
constant U : Type
constant par : U → Prop
constant impar : U → Prop
constant sucessor : U → U

#check ∀ x : U, par x ∨ impar x
#check ∀ y : U, par y → impar (sucessor y)
\end{lstlisting}

        Os dois \lstinline{#check}'s nos retornam \lstinline{Prop}. Quando é utilizado o quantificador universal,
        por padrão devemos dizer o tipo da variável que o acompanha. No exemplo, \lstinline{x : U}. Porém, o Lean é
        esperto o suficiente pra inferir o tipo da variável sozinho, então na maioria dos casos não será um problema omiti-lo.

        Observe as três sentenças:

        \begin{itemize}
            \item $\forall x \ (par(x) \lor impar(x))$
            \item $\forall x \ par(x) \lor impar(x)$
            \item $\forall x \ (par(x)) \lor impar(x)$
        \end{itemize}

        Por uma questão de convenção, as duas últimas sentenças são equivalentes, enquanto a primeira é diferente.
        Neste contexto, estamos lidando com o \textbf{escopo} da variável $x$. A convenção diz, portanto, que o escopo da variável é o menor possível.
        
        Curiosamente, o modo como o Lean lida com escopo é diferente: \lstinline{∀ x : U, par x ∨ impar x} equivale a \lstinline{∀ x : U, (par x ∨ impar x)}.
        Ou seja, o Lean busca o maior escopo possível.

        Quando estamos lidando com quantificadores, a variável que o acompanha é dita \textbf{limitada} (\textit{bound}, em inglês).
        Na expressão $\forall x \ A(x)$, a variável $x$ é limitada.
        Isso significa que o $x$ não representa um valor em si, mas apenas um ``espaço reservado"\space para qualquer outra variável.
        Observe que a expressão $\forall y \ A(y)$ representa exatamente a mesma coisa.

        Uma variável que não é limitada é chamada \textbf{livre}. Por exemplo, considere a expressão $\forall x \ y \le x$.
        Ora, sabemos que $x$ representa uma ``espaço reservado"\space para toda variável do domínio. O que representa $y$, portanto?
        Um elemento específico do domínio. Digamos que o domínio é $\mathbb{N}$ e $y$ representa o elemento zero. Então a expressão é verdadeira.
        Sendo assim, se trocarmos $y$ por $z$ (que representa, neste caso, outro elemento), então não vale a expressão $\forall x \ z \le x$.
        Observe como trocar $y$ por $z$ fez toda a diferença!

        No exemplo anterior, tanto $y$ quanto $z$ são variáveis livres. Quando uma expressão \textbf{não} contém variáveis livres, é chamada sentença.

        Quantificadores também possuem regras de introdução e eliminação. Vejamos:
        
        \begin{center}
            \begin{bprooftree}
                \AxiomC{$A(y)$}
                \RightLabel{\scriptsize $\forall I$}
                \UnaryInfC{$\forall x \ A(x)$}
            \end{bprooftree}
            \begin{bprooftree}
                \AxiomC{$\forall x \ A(x)$}
                \RightLabel{\scriptsize $\forall E$}
                \UnaryInfC{$A(t)$} 
            \end{bprooftree}
        \end{center}

        A regra da esquerda demonstra a \textbf{introdução} do quantificador universal. Ela vale quando $y$ não é livre em nenhuma hipótese não cancelada.
        A intuição neste caso é que, dado que vale $A(y)$ para algum $y$ qualquer, então podemos dizer que vale para todo $y$. Mudamos o nome da variável apenas
        pra que a operação fique mais evidente e intuitiva.

        A regra da direita demonstra a \textbf{eliminação} do quantificador universal. A intuição, neste caso,
        é que, se $A(x)$ vale para todo $x$, então vale para algum $t$ qualquer do domínio.

        Observe a semelhança dessas regras com aquelas relativas à implicação. No caso da introdução da implicação, assumimos $A$ e provamos $B$, então $A \rightarrow B$.
        No caso da introdução do quantificador universal, assumimos $x$ e mostramos $A(x)$, então $\forall x \ A(x)$.
        Para a eliminação da implicação, temos $A \rightarrow B$ e $A$, então temos $B$. Já no caso da eliminação do quantificador, temos $\forall x \ A(x)$ e $y$, então $A(y)$.

        Vejamos um exemplo aplicando essas regras. Observe como derivar $\forall x \ A(x) \land B(x)$ a partir de $\forall x \ A(x)$ e $\forall x \ B(x)$:

        \begin{center}
            \begin{bprooftree}
                \AxiomC{$\forall x \ A(x)$}
                \RightLabel{\scriptsize $\forall E$}
                \UnaryInfC{$A(y)$}
                \AxiomC{$\forall x \ B(x)$}
                \RightLabel{\scriptsize $\forall E$}
                \UnaryInfC{$B(y)$}
                \RightLabel{\scriptsize $\land I$}
                \BinaryInfC{$A(y) \land B(y)$}
                \RightLabel{\scriptsize $\forall I  $}
                \UnaryInfC{$\forall x \ A(x) \land B(x)$}
            \end{bprooftree}
        \end{center}

        Ora, e como representar isso em Lean? Vejamos a introdução:

        \begin{lstlisting}
constant U : Type
constant A : U → Prop

example : ∀ x, A x :=
    assume y,
    show A y, from sorry
\end{lstlisting}

        Estamos mostrando $\forall x \ A(x)$ da seguinte forma: assumimos um $y$ qualquer e provamos $A(y)$.
Neste caso, necessitamos de uma prova de $A(y)$, o que justifica o uso do \lstinline{sorry}.

        Observe, agora, a regra da eliminação:

        \begin{lstlisting}
constant U : Type
constant A : U → Prop
constant h1 : ∀ x, A x
constant t : U

example : A t :=
    h1 t
\end{lstlisting}

        Dado que sabemos que $\forall x \ A(x)$ (por \lstinline{h1}), podemos ``aplicar"\space $t$ e obter $A(t)$.

        Vejamos agora nosso exemplo que deriva $\forall x \ A(x) \land B(x)$ a partir de $\forall x \ A(x)$ e $\forall x \ B(x)$:

        \begin{lstlisting}
constant U : Type
constants A B : U → Prop
constant h1 : ∀ x, A x
constant h2 : ∀ x, B x

example : ∀ x, A x ∧ B x :=
    assume y,
        have h3 : A y, from h1 y,
        have h4 : B y, from h2 y,
    show A y ∧ B y, from and.intro h3 h4
\end{lstlisting}

    \subsection{Quantificador Existencial}

        O quantificador existencial é representado pelo símbolo $\exists$ e representa ``existe algum".
        Seguido de uma variável e uma expressão, ele significa que existe algum elemento no domínio tal que a expressão seja verdadeira.
        Alguns exemplos para o domínio dos naturais:

        \begin{itemize}
            \item $\exists x \ x \times x = x$
            \item $\exists y \ y \le 0$
            \item $\forall x \ \exists y \ par(y) \land y > x$
        \end{itemize}

        Ora, as expressões significam:

        \begin{itemize}
            \item Existe número natural que é igual ao seu quadrado ($0$ e $1$)
            \item Existe número menor do que ou igual a zero (o próprio zero)
            \item Para todo natural, existe um número que é par e maior do que ele
        \end{itemize}

        Como no caso do quantificador universal, a variável que acompanha o quantificador é dita \textbf{limitada}. Por exemplo, $x$ em $\exists x \ A(x)$.
        Variáveis não limitadas são \textbf{livres}.
    \section{Semântica}
    
   A seção anterior abordou a sintaxe da lógica de primeira ordem, ou seja, apresentou os símbolos e estruturas das fórmulas deste sistema. A semântica, por outro lado, define a verdade das fórmulas pela \textbf{interpretação} e \textbf{valoração}.
    
    \subsection{Interpretações}
    
    \subsection{Verdade e modelos}
    
    \subsection{Exemplos}
    
    \subsection{Validação e consequência lógica}
    
    \subsection{Correção e completude}
    

    \section{Dedução Natural}
Como visto na seção anterior devemos estabelecer as regras de inclusão e exclusão
de dedução para os quantificadores e para a igualdade. Listamos elas inicialmente:
\newline \textbf{Quantificador universal:}
 \newline 
 \begin{center}
    \AxiomC{$A(y)$}
    \RightLabel{\scriptsize $\forall I$}
    \UnaryInfC{$\forall x A(x)$}
    \DisplayProof
\end{center}
    Para inserir o quantificador universal devemos ter 
    que a variável $y$ não deve estar atrelada em nenhuma
     hipótese, isto é, em todas as hipóteses não canceladas
      ela não pode ser livre. 
\begin{center}

   
    \AxiomC{$\forall x A(x)$}
    \RightLabel{\scriptsize $\forall E$}
    \UnaryInfC{$A(t)$}
    \DisplayProof   
 \end{center}


 \textbf{Quantificador existencial:}
 \begin{center}
     \AxiomC{$A(t)$}
     \RightLabel{\scriptsize $\exists I$}
     \UnaryInfC{$\exists x A(x)$}
     \DisplayProof

     \AxiomC{$\exists x A(x)$}
        \AxiomC{}
        \UnaryInfC{$A(t)$}
        \alwaysNoLine
        \UnaryInfC{$\vdots$}
        \UnaryInfC{$B$}
        \alwaysSingleLine
        \RightLabel{\scriptsize $\exists E$}
        \BinaryInfC{$B$}
     \DisplayProof
\end{center}

Para retirarmmos o quantificador existencial, a variável $t$ não pode estar livre em $B$, isto é, ela não 
deve estar livre em qualquer hipótese não cancelada.

\textbf{Igualdade:}
\begin{center}
    \AxiomC{}
    \RightLabel{\scriptsize refl}
    \UnaryInfC{$t=t$}
    \DisplayProof

    \AxiomC{$t=s$}
    \RightLabel{\scriptsize sim}
    \UnaryInfC{$s=t$}
    \DisplayProof

    \AxiomC{$t=s$}
    \AxiomC{$s=v$}
    \RightLabel{\scriptsize trans}
    \BinaryInfC{$t=v$}
    \DisplayProof

    \AxiomC{$t=s$}
    \RightLabel{\scriptsize subs}
    \UnaryInfC{$r(t) = r(s)$}
    \DisplayProof

    \AxiomC{$t=s$}
    \AxiomC{$P(t)$}
    \RightLabel{\scriptsize subs}
    \BinaryInfC{$P(s)$}
    \DisplayProof
\end{center}
Iremos passar por cada um destas regras para apresentá-las com o auxílio de exemplos.

\subsection{Quantificador universal}
Como primeiro exemplo segue abaixo uma dedução natural de 
$(\forall x P(x) \to \forall x Q(x)) \to \forall x (P ( x) \to Q (x))$. 
Note que apesar de parecer uma implicação de duas fórmulas idênticas, na premissa a propriedade
 $P$ recebe uma variável e a propriedade $Q$ pode receber outra variável, enquanto que na conclusão
  o valor que $x$ assume é o mesmo para $P$ e $Q$.

\begin{center}
    \AxiomC{}
    \RightLabel{\scriptsize 1}
    \UnaryInfC{$\forall x P(x) \to \forall x Q(x)$}
    \RightLabel{\scriptsize $\forall E$}
    \UnaryInfC{$P(t) \to \forall x Q(x)$}
    \RightLabel{\scriptsize $\forall E$}
    \UnaryInfC{$P(t) \to Q(t)$}
    \RightLabel{\scriptsize $\forall I$}
    \UnaryInfC{$\forall x (P(x) \to Q(x))$}
    \RightLabel{\scriptsize 1}
    \UnaryInfC{$\forall x P(x) \to \forall x Q(x) \to \forall x (P(x) \to Q(x))$}
    \DisplayProof
\end{center}

O primeiro passo será a exclusão da implicação (a implicação principal da fórmula), 
assumindo $\forall x P(x) \to \forall x Q(x)$ como nossa hipótese. Aplicamos uma primeira exclusão do
 universal em $\forall x P(x)$ e novamente aplicamos a exclusão do universal em $\forall x Q(x)$. Note que 
 ao definirmos a variável com o mesmo nome $t$ em ambas as exclusões do universal foi o que permitiu inserir 
 um universal que englobe ambas variáveis, que é o passa seguinte.
\newline Vamos para um outro exemplo, utilizando as hipóteses 
$\forall x (\neg Q(x) \to R(x))$ e $\forall x(P(x) \land \neg Q(x))$ provaremos $\forall x R(x)$:

\begin{center}
    \AxiomC{$\forall x(P(x) \land \neg Q(x))$}
    \RightLabel{\scriptsize $\forall E_1$}
    \UnaryInfC{$P(t) \land \neg Q(t)$}
    \RightLabel{\scriptsize $\land E_2$}
    \UnaryInfC{$\neg Q(t)$}
    \AxiomC{$\forall x (\neg Q(x) \to R(x))$}
    \RightLabel{\scriptsize $\forall E_3$}
    \UnaryInfC{$\neg Q(t) \to R(t)$}
    \RightLabel{\scriptsize $\to E_4$}
    \BinaryInfC{$R(t)$}
    \RightLabel{\scriptsize $\forall I_5$}
    \UnaryInfC{$\forall x R(x)$}
    \DisplayProof
\end{center}
Vamos inicialmente utilizar de nossas hipóteses, em $\forall E_1$ nós executamos a primeira 
exclusão do universal, utilizando a variável $t$ e em seguida executamos a exclusão do $\land$ 
visto que não é necessário o $P(t)$ na dedução. Do outro lado executamos mais uma exclusão do universal
 em $\forall E_3$ e atribuímos novamente a mesma variável $t$. Com $\neg Q(t)$ e $\neg Q(t) \to R(t)$ 
 podemos realizar uma exclusão da implicação em $\to E_4$ e por fim incluímos o universal em $\forall 
 I_5$.

\subsection{Quantificador existencial}
Lembrando a regra de exclusão do existencial, o que fazemos é que com $\exists x A(x)$, supomos um $y$
arbitrário que satisfaça $A(y)$, a partir desta premissa chegamos até $B$, uma fórmula que não contém
$y$ ou qualquer outra variável aberta em alguma hipótese não cancelada, e podemos concluir $B$.
\newline Já a regra da inclusão do existencial, se uma propriedade vale para um $y$ arbitrário, então
existe uma váriavel em que ela é válida.
\newline Iniciemos com uma demosntração simples de que se existe $x$ que satisfaça $A$ e $B$, então 
existe $x$ que satisfaça $A$.

\begin{center}
    \AxiomC{}
    \RightLabel{\scriptsize 1}
    \UnaryInfC{$\exists x (A(x) \land B(x))$}
    \AxiomC{}
    \RightLabel{\scriptsize 2}
    \UnaryInfC{$A(t) \land B(t)$}
    \UnaryInfC{$A(t)$}
    \UnaryInfC{$\exists x A(x)$}
    \RightLabel{\scriptsize 2}
    \BinaryInfC{$\exists x A(x)$}
    \RightLabel{\scriptsize 1}
    \UnaryInfC{$\exists x (A(x) \land B(x)) \to \exists x A(x)$}
    \DisplayProof
\end{center}
A primeira etapa foi a retirada da implicação, passando $\exists x (A(x) \land B(x))$ como uma
hipótese, e utilizando ela aplicamos a regra de exclusão do existencial, criando uma variável
$t$ arbitrária em que $A(x) \land B(x)$ sejá válido, com isso podemos concluir $A(t)$, note que em $2$
não podemos concluir $A(t)$, pois a variável ainda está aberta em $A(t)\land B(t)$, por isso inserimos
o existencial e concluímos $\exists x A(x)$, resultado que queríamos obter.
\newline O próximo exemplo relaciona os quantificadores universal e existencial, se para todo $x$ $A$
é válido, então existe algum $x$ que $A$ sejá válido.
\begin{center}
    \AxiomC{$\forall x A(x)$}
    \UnaryInfC{$A(t)$}
    \UnaryInfC{$\exists x A(x)$}
    \DisplayProof
\end{center}
Note que se $A(x)$ vale para todo $x$, também vale para um $t$ específico e no passo seguinte poderíamos
tanto utilizar a regra da inclusão do universal ou do existencial. Outro comentário relevante é que não
necessariamente precisávamos concluir o existencial utilizando a mesma variável $x$.
\newline Vamos provar mais uma relação entre os quantificadores, iremos que provar que se para todo $x$ 
não vale $A$, então não existe $x$ tal que $A$ valha:
\begin{center}
    \AxiomC{}
    \RightLabel{\scriptsize 2}
    \UnaryInfC{$\exists x A(x)$}
    \AxiomC{}
    \RightLabel{\scriptsize 3}
    \UnaryInfC{$A(t)$}
    \AxiomC{}
    \RightLabel{\scriptsize 1}
    \UnaryInfC{$\forall x \neg A(x)$}
    \UnaryInfC{$ \neg A(t)$}
    \BinaryInfC{$\bot $}
    \RightLabel{\scriptsize 3}
    \BinaryInfC{$\bot$}
    \RightLabel{\scriptsize 2}
    \UnaryInfC{$\neg \exists x A(x)$}
    \RightLabel{\scriptsize 1}
    \UnaryInfC{$\forall x \neg A(x) \to \neg \exists x A(x)$}
    \DisplayProof
\end{center}
Para a nossa dedução a primeira etapa foi desmontar a implicação, passando $\forall x \neg A(x)$ como
uma de nossas hipóteses, em seguida, como temos uma negação devemos chegar até ao falso. Utilizamos 
$\exists x A(x)$ como mais uma de nossas hipóteses e aplicamos a regra de exclusão do existencial,
dessa forma obtemos no mesmo ramo $A(t)$ e $\neg A(t)$, obtendo a contradição que estávamos procurando.
\subsection{Igualdade}
TO DO 
\subsection{Exercícios}
TO DO
\subsection{Dedução natural no LEAN}
No Lean a dedução ocorre de forma similar a dedução em primeira ordem, apenas devemos utilizar de novos símbolos e das 
regras de exclusão e inclusão dos quantificadores. Os símbolos $\exists $ e $\forall $ são escritos no Lean como \textbackslash exist 
e  \textbackslash all. Para a regra de exclusão do universal apenas passamos para uma proposição $\forall x A(x)$ uma letra,
por exemplo $t$, para termos $A(t)$. Para a inclusão do universal, devemos assumir uma letra, por exemplo $t$, utilizando o
"assume" e com esta letra livre de qualquer hipótese provamos $A(t)$, dessa forma o Lean é capaz de inferir $\forall x A(x)$.
\newline Vamos utilizando o Lean provar o nosso primeiro exemplo do quantificador universal, $\forall xP(x) \to \forall x Q(x) \to \forall x(P(x) \land Q(x))$:
\begin{lstlisting}
 example : $\forall$x P(x) $\to$ $\forall$x Q(x) $\to \forall$x (P(x) $\land$ Q (x)) :=  
 assume h$_1$ : $\forall$x P(x),
 assume h$_2$ : $\forall$x Q(x),
 assume t,
 have h$_3$ : P(t), from h$_1$ t,
 have h$_4$ : Q(t), from h$_2$ t,
 show P(t) $\land$ Q(t), from and.intro h$_3$ h$_4$ 
\end{lstlisting}
Note que na linha 4 utilizamos da inclusão do universal, assumimos um $t$ e desejamos provar $P(t) \land Q(t)$ 
sem que $t$ possua qualquer restrição e nas linhas 5 e 6 utilizamos a exclusão do universal, temos fórmulas do
tipo $\forall x P(x)$ e passamos a letra $t$, obtendo $P(t)$.
\newline Para as regras do existencial, na exclusão utilizamos "exists.elim" seguida por uma proposição do tipo $\forall x A(x)$,
para provarmos $B$, devemos provar $ A y \to B$, ou seja, assumimos um $y$ e $A y$ e chegamos até $B$, concluindo assim a exclusão
do existencial. Para a inclusão do existencial utilizamos "exists.intro", devemos passar uma letra, por exemplo $t$, e uma prova
de que $A(t)$ vale.