\chapter{Funções}

No final do século XIX, diversas áreas da matemática 
estudam consistentemente conjuntos, relações, já apresentadas
nesse texto, e funções. Neste capítulo, debruçaremos nossa 
atenção nas propriedades dessa terceira área. 

Entende-se que uma função $f$ é um mapeamento entre um 
domínio $X$ e um domínio $Y$, este que será conhecido como 
contra domínio, posteriormente. Entretanto, para teóricos 
de conjuntos, esses domínios são considerados conjuntos. 
Vimos que Lean é uma linguagem baseada em Tipos. Dessa forma, 
estabelece-se uma diferença clara entre os Tipos, que são
os Domínios, enquanto podemos dizer que $A$ é do tipo conjunto
de $X$ (Lembre, em Lean: $A : set~ X$). 

Entretanto, a visão da função como mapeamento entre conjuntos
é comum entre os matemáticos e será considerada nesse texto,
fazendo as devidas comparações com a linguagem de referência, 
o Lean. 


