\chapter{Funções}

Desde o final do século XIX, diversas áreas da matemática 
estudam consistentemente \textit{\hyperlink{chapter.4}{conjuntos}}, 
\textit{\hyperlink{chapter.5}{relações}}, já apresentadas 
nesse texto, e funções. Neste capítulo, debruçaremos nossa 
atenção nas propriedades dessa terceira área. 

Entende-se que uma função $f$ é um mapeamento entre um 
domínio $X$ e um domínio $Y$, este que será conhecido como 
contradomínio, posteriormente. Entretanto, para teóricos 
de conjuntos, esses domínios são simplesmente considerados 
conjuntos. Vimos que \textit{\hyperlink{chapter.2}{Lean}} é uma 
linguagem baseada em Tipos. Dessa forma, estabelece-se uma 
diferença clara entre o Domínio $X$, caracterizado pelos Tipos, e
o Conjunto $A$, que é do tipo (sub)conjunto de $X$. Em Lean: 
\begin{lstlisting}
    variable X : Type
    variable A : set X    
\end{lstlisting}

Entretanto a visão da função como mapeamento entre conjuntos
é comum entre os matemáticos e será considerada nesse texto,
fazendo as devidas comparações com a linguagem de referência, 
o Lean. 

\section{O Conceito de Função}

Considere dois conjuntos quaisquer $X$ e $Y$ e um mapeamento $f$ 
do conjunto $X$ para o conjunto $Y$. Se $f$ atribui um e apenas 
um valor para cada elemento de $X$, dizemos que $f$ é uma função total, e
escrevemos $f: X \to Y$.  Chamamos $X$ de domínio de $f$, enquanto 
$Y$ é o contradomínio de $f$ e $\forall x, (x \in X  \Rightarrow f(x) \in Y)$. 
Nesse sentido, $\forall x_1 \in X, \forall x_2 \in X, x_1 = x_2 \Rightarrow f(x_1) = f(x_2)$.
Uma função também pode ser parcial quando ela não é definida para alguns valores 
do domínio. Como toda função parcial é total ao restringir o domínio, trateremos 
nesse texto das funções totais. 

A maneira mais simples de se representar uma função é escrevê-la explicitamente 
para cada elemento do domínio. Por exemplo, podemos escrever as seguintes expressões:

\begin{itemize}
    \item Seja $f: \mathbb{N} \to \mathbb{N}$ definida por 
    $f(n) = 2\cdot n + 1$
    \item Seja $g : \mathbb{N} \to \mathbb{R}$ definida por 
    $g(n) = \frac{n}{n+1}$
    \item Seja $h : \mathbb{R} \to \{0,1\}$ definida por 
    $$\left \{ \begin{array}{c}
    h(x) = 1 ~if~x \in \mathbb{Q} \\
    h(x) = 0 ~ if ~ x \not \in \mathbb{Q} \\
    \end{array}
    \right. $$
 \end{itemize}

A questão que se levanta é: o que torna uma expressão explícita legítima? 
Neste momento, deixaremos essa questão de lado e notaremos que a matemática 
é confortável com diversos tipos de definições, como, por exemplo, a definição 
de $h(x)$ no exemplo acima. Nesse mesmo exemplo, não fica claro em como podemos 
definir algoritmicamente se um número real de entrada é um número racional ou não.
Porém, esse não é o objetivo do capítulo.

Note que a escolha das variáveis $x$ e $n$ são arbitrárias. 
Isto nos leva à definição no capítulo de \textit{\hyperlink{chapter.2}{FOL}} de 
variável ligada (\textit{bound}), visto que se renomearmos $x$ com $y$, os valores
continuam os mesmos. 

Lógicos frequentemente utilizam a notação $\lambda x e(x)$ para denotar a função
que mapeia $x$ para $e(x)$. Essa notação chama-se \textit{notação lambda} e pode 
ser usada da seguinte forma: $f = \lambda x(x + 1)$, que significa $f(x) = x + 1$. 
Essa notação é mais interessante para cientistas da computação e lógicos do que 
propriamente para matemáticos. Em Lean, definimos da seguinte forma: 

\begin{lstlisting}
    variables X Y : Type
    variable f : X → Y    
\end{lstlisting}

Lembre-se que diferenciamos o tipo $X$ do tipo $set~X$ em Lean.

\section{Primeiras Definições}

\textit{Definição:} Seja um conjunto $X$. A função identidade de $A$ é a função 
$i_A : A \rightarrow A$ definida para todos os valores $x \in A$ tal que $i_A(x) = x$.

\textit{Definição:} Sejam $f : X \rightarrow Y$ e $g : Y \rightarrow Z$ funções. 
Defina $k : X \rightarrow Z$ por $k(x) = g(f(x))$. A função $k$ é chamada de composição
de $f$ e $g$ ou $f$ composta com $g$ e é escrita $g \circ f$. Desta forma, para cada elemento, 
primeiro  avaliamos $f(x) \in Y$ e depois avaliamos $g(f(x)) \in Z$. 

\begin{lstlisting}
    namespace hidden
        variables {X Y Z : Type}

        def comp (f : Y → Z) (g : X → Y) : X → Z :=
        λx, f (g x)

        infixr  ` ∘ ` := comp
        #check @comp

        def id (x : X) : X := x
        
    end hidden
\end{lstlisting}

\textit{Definição:} Considere $f,g : X \to Y$. Dizemos que essas funções são iguais, 
quando para todos os valores do domínio $X$, a correspondência no contradomínio $Y$ é a 
mesma. Em lógica simbólica, $\forall x (x \in X \to (f(x) = g(x)) \iff f = g $. 

Obseve que em termos de lógica formal de tipos, poderíamos reescrever como 
$\forall x : X (f(x) = g(x)) \iff f = g$. Escrevemos essa equivalência entre lógica e funções 
utilizando a extensionalidade das funções, muito semelhante à descrita no capítulo de 
\textit{\hyperlink{chapter.5}{conjuntos}}. Por exemplo, se $f, g : \mathbb{R} \to \mathbb{R}$ 
definidas por $f(x) = x + 1$ e $g(x) = x + 1$, então $f = g$, pois para cada valor de $x$, vale 
a comutativade da soma. 

\textit{Observação: colocamos dentro de um namespace hidden, pois id já é definido.}

\begin{lstlisting}
    #check @id
\end{lstlisting} 

\newtheorem{theorem}{Proposição}

\begin{theorem}
    Para todo $f: X \to Y$, $f \circ i_X = f$ e $i_Y \circ f = f$
\end{theorem}
\begin{proof}
    Seja $x$ um elemento qualquer de $X$. Então $(f \circ i_X)(x) = f(i_X(x)) = f(x)$
     e $(i_Y \circ f)(x) = i_Y(f(x)) = f(x)$, o que mostra a igualdade. 
\end{proof}

\textit{Definição:} Suponha $f:X \to Y$ e $g : Y \to X$ satisfaz $g \circ f = i_X$. Assim, 
$g(f(x)) = i_X(x) = x$, para todos os elementos de $X$. Neste caso $g$ é dita inversa à esquerda
de $f$ e $f$ é dita inversa à direita de $g$.

\vspace{3mm}

\textit{Exemplos}

\begin{itemize}
    \item Defina $f,g : \mathbb{R} \to \mathbb{R}$ por $f(x) = x + 1$ e $g(x) = x - 1$. Então $g$ 
    é inversa à direita e à esquerda de $f$ e vice-versa.
    \item $\mathbb{R}^{+}$ denota os reais não negativos. Defina $f : \mathbb{R} \to \mathbb{R}^{+}$ 
    por $f(x) = x^2$ e defina $g : \mathbb{R}^2 \to \mathbb{R}$ por $g(x) = \sqrt{x}$. Então 
    $f(g(x)) = (\sqrt{x})^2 = x$, para todo $x$ no domínio de $g$. Então $f$ é inversa à esquerda de $g$
    e $g$ inversa à direita de $f$. Por outro lado, $g(f(x)) = \sqrt{x^2} = |x|$. Logo $g$ não é inversa à 
    esqueda de $f$ e $f$ não é inversa à direita de $g$. 
\end{itemize}

\begin{theorem}
    Suponha que $f : X \to Y $ tem inversa à esquerda, $h : Y \to X $ e uma inversa à direita, $k : Y \to X $.
    Então $h = k$. 
\end{theorem}
\begin{proof}
    Seja $y \in Y$. $h(f(k(y))) = k(y)$, pois $h$ é uma inversa à esquerda de $f$. 
    Por outro lado, $f(k(y)) = y$ e, portanto $h(f(k(y))) = h(y)$. Assim $k(y) = h(y) $ e as funções são iguais. 
\end{proof}
