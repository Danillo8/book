\chapter{Funções}

Desde o final do século XIX, diversas áreas da matemática 
estudam consistentemente \textit{\hyperlink{chapter.4}{conjuntos}}, 
\textit{\hyperlink{chapter.5}{relações}}, já apresentadas 
nesse texto, e funções. Neste capítulo, debruçaremos nossa 
atenção nas propriedades dessa terceira área. 

Entende-se que uma função $f$ é um mapeamento entre um 
domínio $X$ e um domínio $Y$, este que será conhecido como 
contradomínio, posteriormente. Entretanto, para teóricos 
de conjuntos, esses domínios são simplesmente considerados 
conjuntos. Vimos que \textit{\hyperlink{chapter.2}{Lean}} é uma 
linguagem baseada em Tipos. Dessa forma, estabelece-se uma 
diferença clara entre o Domínio $X$, caracterizado pelos Tipos, e
o Conjunto $A$, que é do tipo (sub)conjunto de $X$. Em Lean: 
\begin{lstlisting}
    variable X : Type
    variable A : set X    
\end{lstlisting}

Entretanto, a visão da função como mapeamento entre conjuntos
é comum entre os matemáticos e será considerada nesse texto,
fazendo as devidas comparações com a linguagem de referência, 
o Lean. 

\section{O Conceito de Função}

Considere dois conjuntos quaisquer $X$ e $Y$ e um mapeamento $f$ 
do conjunto $X$ para o conjunto $Y$. Se $f$ atribui um e apenas 
um valor para cada elemento de $X$, dizemos que $f$ é uma função, e
escrevemos $f: X \to Y$. Chamamos $X$ de domínio de $f$, enquanto 
$Y$ é o contradomínio de $f$ e $\forall x, (x \in X  \Rightarrow f(x) \in Y)$. 

As funções podem ser escritas explicitamente para cada elemento 
do domínio. Por exempl, podemos escrever as seguintes expressões:

\begin{itemize}
    \item Seja $f: \mathbb{N} \to \mathbb{N}$ definida por 
    $f(n) = 2\cdot n + 1$
    \item Seja $g : \mathbb{N} \to \mathbb{R}$ definida por 
    $g(n) = \frac{n}{n+1}$
    \item Seja $h : \mathbb{R} \to \{0,1\}$ definida por 
    $$\left \{ \begin{array}{c}
    h(x) = 1 ~if~x \in \mathbb{Q} \\
    h(x) = 0 ~ if ~ x \not \in \mathbb{Q} \\
    \end{array}
    \right. $$
 \end{itemize}

 A questão que se levanta é o que torna uma expressão explícita 
 legítima? Nesse momento, não discutiremos esse problema e 
 aceitaremos que a matemática é confortável com esse tipo de definição. 
 Todavia, isso está em desacordo com a ideia de funções serem 
 objetos que são computáveis. No último exemplo, não fica claro 
 em como, algorimiticamente, dado um real como entrada, decidiremos
 se ele é racional ou não é. Em um capítulo a frente, voltaremos a 
 essa questão novamente. 
