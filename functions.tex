\chapter{Funções}

Desde o final do século XIX, diversas áreas da matemática
estudam consistentemente \textit{\hyperlink{chapter.4}{conjuntos}},
\textit{\hyperlink{chapter.5}{relações}}, já apresentadas
nesse texto, e funções. Neste capítulo, debruçaremos nossa
atenção nas propriedades dessa terceira área.

Entende-se que uma função $f$ é um mapeamento entre um
domínio $X$ e um domínio $Y$, este que será conhecido como
contradomínio, posteriormente. Entretanto, para teóricos
de conjuntos, esses domínios são simplesmente considerados
conjuntos. Vimos que \textit{\hyperlink{chapter.2}{Lean}} é uma
linguagem baseada em Tipos. Dessa forma, estabelece-se uma
diferença clara entre o Domínio $X$, caracterizado pelos Tipos, e
o Conjunto $A$, que é do tipo (sub)conjunto de $X$. Em Lean:
\begin{lstlisting}
variable X : Type
variable A : set X
\end{lstlisting}

Entretanto a visão da função como mapeamento entre conjuntos
é comum entre os matemáticos e será considerada nesse texto,
fazendo as devidas comparações com a linguagem de referência,
o Lean.

\section{O Conceito de Função}

Considere dois conjuntos quaisquer $X$ e $Y$ e um mapeamento $f$
do conjunto $X$ para o conjunto $Y$. Se $f$ atribui um e apenas
um valor para cada elemento de $X$, dizemos que $f$ é uma função total, e
escrevemos $f: X \to Y$.  Chamamos $X$ de domínio de $f$, enquanto
$Y$ é o contradomínio de $f$ e $\forall x, (x \in X  \Rightarrow f(x) \in Y)$.
Nesse sentido, $\forall x_1 \in X, \forall x_2 \in X, x_1 = x_2 \Rightarrow f(x_1) = f(x_2)$.
Uma função também pode ser parcial quando ela não é definida para alguns valores
do domínio. Como toda função parcial é total ao restringir o domínio, trateremos
nesse texto das funções totais.

A maneira mais simples de se representar uma função é escrevê-la explicitamente
para cada elemento do domínio. Por exemplo, podemos escrever as seguintes expressões:

\begin{itemize}
    \item Seja $f: \mathbb{N} \to \mathbb{N}$ definida por
    $f(n) = 2\cdot n + 1$
    \item Seja $g : \mathbb{N} \to \mathbb{R}$ definida por
    $g(n) = \frac{n}{n+1}$
    \item Seja $h : \mathbb{R} \to \{0,1\}$ definida por
    $$\left \{ \begin{array}{c}
    h(x) = 1 ~if~x \in \mathbb{Q} \\
    h(x) = 0 ~ if ~ x \not \in \mathbb{Q} \\
    \end{array}
    \right. $$
 \end{itemize}

A questão que se levanta é: o que torna uma expressão explícita legítima?
Neste momento, deixaremos essa questão de lado e notaremos que a matemática
é confortável com diversos tipos de definições, como, por exemplo, a definição
de $h(x)$ no exemplo acima. Nesse mesmo exemplo, não fica claro em como podemos
definir algoritmicamente se um número real de entrada é um número racional ou não.
Porém, esse não é o objetivo do capítulo.

Note que a escolha das variáveis $x$ e $n$ são arbitrárias.
Isto nos leva à definição no capítulo de \textit{\hyperlink{chapter.4}{FOL}} de
variável ligada (\textit{bound}), visto que se renomearmos $x$ com $y$, os valores
continuam os mesmos.

Lógicos frequentemente utilizam a notação $\lambda x e(x)$ para denotar a função
que mapeia $x$ para $e(x)$. Essa notação chama-se \textit{notação lambda} e pode
ser usada da seguinte forma: $f = \lambda x(x + 1)$, que significa $f(x) = x + 1$.
Essa notação é mais interessante para cientistas da computação e lógicos do que
propriamente para matemáticos. Em Lean, definimos da seguinte forma:

\begin{lstlisting}
variables X Y : Type
variable f : X → Y
\end{lstlisting}

Lembre-se que diferenciamos o tipo $X$ do tipo $set~X$ em Lean.

\section{Primeiras Definições}

\theoremstyle{definition}
%\newtheorem{definition}{Definição}[section]

\theoremstyle{definition}
\newtheorem{example}{Exemplo}[section]

\theoremstyle{plain}
%\newtheorem{theorem}{Proposição}[section]

\theoremstyle{plain}
\newtheorem{corollary}{Corolário}[section]

\begin{definition}
    \label{def1}
    Seja um conjunto $X$. A função identidade de $A$ é a função
    $i_A : A \rightarrow A$ definida para todos os valores $x \in A$ tal que $i_A(x) = x$.
\end{definition}

\begin{definition}
    \label{def2}
    Sejam $f : X \rightarrow Y$ e $g : Y \rightarrow Z$ funções.
    Defina $k : X \rightarrow Z$ por $k(x) = g(f(x))$. A função $k$ é chamada de composição
    de $f$ e $g$ ou $f$ composta com $g$ e é escrita $g \circ f$. Desta forma, para cada elemento,
    primeiro  avaliamos $f(x) \in Y$ e depois avaliamos $g(f(x)) \in Z$.
\end{definition}

Podemos ver em Lean as definições de composição e de identidade. Note que estamos em um namespace
hidden para que não haja conflito de definições.

\begin{lstlisting}
namespace hidden
    variables {X Y Z : Type}

    def comp (f : Y → Z) (g : X → Y) : X → Z :=
    λx, f (g x)

    infixr  ` ∘ ` := comp

    #check @comp

    def id (x : X) : X := x

    #check @id

end hidden
\end{lstlisting}

\begin{definition}
    \label{def3}
    Considere $f,g : X \to Y$. Dizemos que essas funções são iguais,
    quando para todos os valores do domínio $X$, a correspondência no contradomínio $Y$ é a
    mesma. Em lógica simbólica, $\forall x (x \in X \to (f(x) = g(x)) \iff f = g $.
\end{definition}

Obseve que em termos de lógica formal de tipos, poderíamos reescrever como
$\forall x : X (f(x) = g(x)) \iff f = g$. Escrevemos essa equivalência entre lógica e funções
utilizando a extensionalidade das funções, muito semelhante à descrita no capítulo de
\textit{\hyperlink{chapter.5}{conjuntos}}. Por exemplo, se $f, g : \mathbb{R} \to \mathbb{R}$
definidas por $f(x) = x + 1$ e $g(x) = 1 + x$, então $f = g$, pois para cada valor de $x$, vale
a comutativade da soma.

Em lean, o comando \lstinline{funext} (de "function extensionality") prova a igualdade de funções.

\begin{lstlisting}
variables {X Y : Type}

example (f g : X → Y) (h : ∀ x, f x = g x) : f = g :=
    funext h
\end{lstlisting}

\begin{theorem}
    \label{prop1}
    Para todo $f: X \to Y$, $f \circ i_X = f$ e $i_Y \circ f = f$
\end{theorem}
\begin{proof}
    Seja $x$ um elemento qualquer de $X$. Então $(f \circ i_X)(x) = f(i_X(x)) = f(x)$
     e $(i_Y \circ f)(x) = i_Y(f(x)) = f(x)$, o que mostra a igualdade.
\end{proof}

No Lean, podemos mostrar essa proposição da seguinte forma. Para algumas funções, será necessário
escrevermos \lstinline{open function}.

\begin{lstlisting}
variables {X Y Z W : Type}

lemma left_id (f : X → Y) : id ∘ f = f := rfl

lemma right_id (f : X → Y) : f ∘ id = f := rfl

theorem comp.assoc (f : Z → W) (g : Y → Z) (h : X → Y) :
    (f ∘ g) ∘ h = f ∘ (g ∘ h) := rfl

\end{lstlisting}

\begin{definition}
    \label{def4}
    Suponha $f:X \to Y$ e $g : Y \to X$ satisfaz $g \circ f = i_X$. Assim,
    $g(f(x)) = i_X(x) = x$, para todos os elementos de $X$. Neste caso $g$ é dita inversa à esquerda
    de $f$ e $f$ é dita inversa à direita de $g$. Quando $g$ é inversa à direita e é inversa à esquerda
    de $f$, então $g$ é dita simplesmente inversa de $f$.
\end{definition}

\begin{example}
    \label{ex1}
    Defina $f,g : \mathbb{R} \to \mathbb{R}$ por $f(x) = x + 1$ e $g(x) = x - 1$. Então $g$
    é inversa à direita e à esquerda de $f$ e vice-versa.
\end{example}
\begin{example}
    \label{ex2}
    $\mathbb{R}^{+}$ denota os reais não negativos. Defina $f : \mathbb{R} \to \mathbb{R}^{+}$
    por $f(x) = x^2$ e defina $g : \mathbb{R}^2 \to \mathbb{R}$ por $g(x) = \sqrt{x}$. Então
    $f(g(x)) = (\sqrt{x})^2 = x$, para todo $x$ no domínio de $g$. Então $f$ é inversa à esquerda de $g$
    e $g$ inversa à direita de $f$. Por outro lado, $g(f(x)) = \sqrt{x^2} = |x|$. Logo $g$ não é inversa à
    esquerda de $f$ e $f$ não é inversa à direita de $g$.
\end{example}

\begin{theorem}
    \label{prop2}
    Suponha que $f : X \to Y $ tem inversa à esquerda, $h : Y \to X $ e uma inversa à direita, $k : Y \to X $.
    Então $h = k$.
\end{theorem}
\begin{proof}
    Seja $y \in Y$. $h(f(k(y))) = k(y)$, pois $h$ é uma inversa à esquerda de $f$.
    Por outro lado, $f(k(y)) = y$ e, portanto $h(f(k(y))) = h(y)$. Assim $k(y) = h(y) $ e as funções são iguais.
\end{proof}

Essa proposição pode também ser vista em Lean. Considerarei a prova utilizando táticas. O leitor pode estudar
aquela que sentir mais confortável. Note que neste exemplo, já foi necessário o \lstinline{namespace function},
visto que \lstinline{left_inverse} e \lstinline{right_inverse} já estão predefinidas.

\begin{lstlisting}
open function

variables {X Y Z: Type}

-- Term and Calc Mode
example (f: X → Y) (h: Y → X) (k: Y → X)
    (hf: left_inverse h f) (fk: right_inverse k f): h = k :=
    have H: ∀ ( x : Y ), h x = k x, from
        assume x,
        have h1: h (f (k x)) = k x , from calc
            h ( f (k x)) = k x: by apply hf,
        have h2: h (f (k x)) = h x, from calc
            h (f (k x)) = h x: by rw fk,
        show h x = k x, from eq.trans (eq.symm h2) h1,
    show h = k, from funext H

-- Tatics Mode
example (f: X → Y) (h: Y → X) (k: Y → X)
    (hf: left_inverse h f) (fk: right_inverse k f): h = k :=
    begin
        apply funext,
        assume x,
        have hf: h (f (k x)) = k x, by apply hf,
        rw ←hf,
        rw fk,
    end
\end{lstlisting}

\begin{theorem}
    \label{prop3}
    Seja $f: X \to Y$. Se a inversa de $f$ existe, então ela é única. Isto é, se $g_1, g_2: Y \to X$ são inversas
    de $f$, então $g_1 = g_2$
\end{theorem}
\begin{proof}
    Sabemos que $g_1$ é inversa à esquerda de $f$. Então,
    $$g_1(f(g_2(x))) = g_2(x), ~para~todos~os~valores~de~x.$$
    Também, $g_2$ é inversa de $f$. Então,
    $$g_1(f(g_2(x))) = g_1(x), ~para~todos~os~valores~de~x.$$ Logo $g_1 = g_2$.
\end{proof}

Quando a inversa de $f$ existe, então, podemos escrevê-la como $f^{-1}$. Dada a Definição \hyperlink{def4},
podemos afirmar que $(f^{-1})^{-1} = f$.

Observe que uma função pode possuir mais de uma função inversa à esquerda ou mais de uma função inversa
à direita. Ainda, quando uma função possui mais de uma função inversa à esquerda, ela não possui inversa
à direita. Se ela possuisse, ela seria igual a todas as funções inversas à esquerda pela Proposição \ref{prop2},
portanto elas seriam iguais, o que é uma contradição, por hipótese. O mesmo vale para inversas à direita.

\begin{theorem}
     \label{prop4}
     Seja $f: X \to Y $ e $g: Y \to Z$. Se $h: Y \to X$ e $k: Z \to Y$ são inversas à esqueda de $f$ e $g$,
     respectivamente, então $h \circ k$ é inversa à esquerda de $g \circ f$. O mesmo vale quando substituimos
     esquerda por direita na proposição.
\end{theorem}

\begin{proof}
 $(h \circ k)\circ(g \circ f)(x) = h(k(g(f(x)))) = h(f(x)) = x $. A demonstração quando substituimos a proposição
 de esquerda para direita é análoga e é deixada como exercício ao leitor.
\end{proof}

\begin{corollary}
    \label{cor1}
    Se $f: X \to Y$ e $g: Y \to Z$ possuem inversas, então $(f \circ g)^{-1}$ existe e
    $(f \circ g)^{-1} = f^{-1} \circ g^{-1}$.
\end{corollary}

\begin{proof}
    Segue diretamente da Proposição \ref{prop4} e Proposição \ref{prop2}.
\end{proof}

\section{Funções Injetiva, Sobrejetiva e Bijetiva}

\begin{definition}[Função Injetiva]
    \label{def5}
    Também conhecida como função injetora, é uma função em que elementos distintos do domínio são
    mapeados para elementos diferentes do contradomínio. Dessa forma, seja $f: X \to Y$. Se
    $\forall x_1 \in X, \forall x_2 \in X, x_1 \neq x_2 \Rightarrow f(x_1) \neq f(x_2) $, f é injetiva.
    A definição pode ser feita pela contrapositiva dessa afirmação, também. A definição pela contrapositiva
    será bastante utilizada nas demonstrações.
\end{definition}

\begin{definition}[Função Sobrejetiva]
    \label{def6}
    Também conhecida como função bijetora, é uma função em que todos os elementos do contradomínio
    estão na imagem da função. Dessa forma, seja $f: X \to Y$. Se $\forall y \in Y, \exists x \in X, f(x) = y$,
    f é sobrejetiva.
   \end{definition}

\begin{definition}[Função Bijetiva]
    \label{def7}
    Também conhecida como bijeção ou correspondência um a um, é uma função simultaneamente injetiva e
    sobrejetiva.
\end{definition}

Em Lean, essas definições são descritas no \lstinline{namespace function}.

\begin{example}
    Considere $X = {1,2,3,4}$ e $Y = \{John, Paul, George, Ringo\}$. É possível construir uma função bijetiva
    entre o conjunto $X$ e o conjunto $Y$.
\end{example}

\begin{lstlisting}
variables {X Y: Type}

def injective (f : X → Y) : Prop :=
∀ x₁ x₂, f x₁ = f x₂ → x₁ = x₂

def surjective (f : X → Y) : Prop :=
∀ y, ∃ x, f x = y

def bijective (f : X → Y) := injective f ∧ surjective f
\end{lstlisting}

\begin{theorem}
    \label{prop5}
    Seja $f : X \to Y$.
    \renewcommand{\labelenumi}{\Roman{enumi}}
    \begin{enumerate}
        \item Se $f$ possui inversa à esquerda, $f$ é injetiva.
        \item Se $f$ possui inversa à direita, $f$ é sobrejetiva.
        \item Se $f$ possui inversa, $f$ é bijetiva.
    \end{enumerate}
\end{theorem}
\begin{proof}
    Para provar I), suponha que $f(x_1) = f(x_2)$ e que $g$ é inversa à esquerda
    de $f$. Assim $g(f(x_1)) = x_1$ e $g(f(x_2)) = x_2)$. Portanto $x_1 = x_2$ e
    está provado. Para provar II), considere $h$ inversa à direita de $f$. Seja
    $y \in Y$ e $x = h(y)$. Então $f(x) = f(h(y)) = y$. A terceira sai diretemente
    das duas anteriores.
\end{proof}

\begin{lstlisting}
open function

variables {X Y : Type}

theorem inj_of_left_inverse {g : Y → X} {f : X → Y} :
left_inverse g f → injective f :=
    assume h, assume x₁ x₂, assume feq,
    calc x₁ = g (f x₁) : by rw h
        ... = g (f x₂) : by rw feq
        ... = x₂       : by rw h

theorem surj_of_right_inverse {g : Y → X} {f : X → Y} :
right_inverse g f → surjective f :=
    assume h, assume y,
    let  x : X := g y in
    have f x = y, from calc
        f x  = (f (g y))    : rfl
        ... = y            : by rw [h y],
    show ∃ x, f x = y, from exists.intro x this
\end{lstlisting}

\begin{theorem}
    \label{prop6}
    Seja $f: X \to Y$
    \renewcommand{\labelenumi}{\Roman{enumi}}
    \begin{enumerate}
        \item Se $X$ é não vazio e $f$ é injetiva, então $f$ possui inversa à esquerda.
        \item Se $f$ é sobrejetiva, então $f$ possui inversa à direita.
        \item Se $X$ é não vazio e $f$ é bijetiva, então $f$ possui inversa.
    \end{enumerate}
\end{theorem}

\begin{proof}
    Seja $\hat{x} \in X$. Defina uma função $g: Y \to X$ com $g(y) = x$, tal que $f(x) = y$,
    caso exista esse $x$. Caso ele não exista, $g(y) = \hat{x}$. Agora, suponha que
    $g(f(x)) = x'$. Pela definição de $g$, $f(x) = y$, logo $g(y) = x$. Assim, $f(x) = f(x')$.
    Como $f$ é injetiva $x = x'$.

    Para a segunda afirmação, defina $h: Y \to X$, onde, para cada $y \in Y$, escolha um elemento
    $x \in X$, tal que $h(y) = x$. Note que a sobrejetividade garante a existência desse elemento,
    mas não garante a unicidade. Então $f(h(y)) = f(x) = y$, por definição de $h$.

    A fim de provar a terceira, basta as demonstrações das anteriores.
\end{proof}

Algumas observações sobre essa demonstração são importantes. Ao definir $g$ na primeira parte,
precisa-se decidir se $x \in X$ existe tal que $f(x) = y$. Isso pode não ser algoritmicamente
feito, logo $g$ poderia não ser computável. Na construção de $h$, a prova requere que haja
uma escolha de valor de $x$ entre os possíveis candidatos. Isto é uma versão do
\href{https://pt.wikipedia.org/wiki/Axioma_da_escolha#Enunciado}{\textit{axioma da escolha}}.
Este axioma foi muito debatido no século XX, mas hoje já é comum para demonstrações. O paradoxo
de Banach-Tarski é um argumento contra o axioma.

Utilizando conceitos e resultados da seção anterior, podemos provar a seguinte proposição.

\begin{theorem}
    Seja $f: X \to Y$ e $g: Y \to Z$.
    \renewcommand{\labelenumi}{\Roman{enumi}}
    \begin{enumerate}
        \item Se $f$ e $g$ são injetivas, $g \circ f$ também será.
        \item Se $f$ e $g$ são sobrejetivas, $g \circ f$ também será.
    \end{enumerate}
\end{theorem}

\begin{proof}
    Basta aplicarmos a Proposição \ref{def4} e definirmos $h$ e $k$ como
    inversas à esquerd ade $f$ e $g$ respefticamente. Logo $(h \circ k)$
    é inversa à esquerda de  $(g \circ f)$ e, portanto, ela é injetiva.

    O mesmo vale para a segunda afirmação.
\end{proof}

\begin{lstlisting}
open function

namespace hidden
    variables {X Y Z : Type}

    theorem injective_comp {g : Y → Z} {f : X → Y}
    (Hg : injective g) (Hf : injective f) :
    injective (g ∘ f) :=
        assume x₁ x₂,
        assume : (g ∘ f) x₁ = (g ∘ f) x₂,
        have f x₁ = f x₂, from Hg this,
        show x₁ = x₂, from Hf this

    theorem surjective_comp {g : Y → Z} {f : X → Y}
        (hg : surjective g) (hf : surjective f) :
    surjective (g ∘ f) :=
    begin
        assume z,
        apply exists.elim (hg z),
        assume y (hy: g y = z),
        apply exists.elim (hf y),
        assume x (hx: f x = y),
        rw ←hx at hy,
        apply exists.intro x hy
    end

    theorem bijective_comp {g : Y → Z} {f : X → Y}
        (hg : bijective g) (hf : bijective f) :
        bijective (g ∘ f) :=
    have ginj : injective g, from hg.left,
    have gsurj : surjective g, from hg.right,
    have finj : injective f, from hf.left,
    have fsurj : surjective f, from hf.right,
    and.intro (injective_comp ginj finj)
                (surjective_comp gsurj fsurj)
end hidden
\end{lstlisting}

\begin{example}
    Considere $f: \mathbb{N} \to Y$, tal que $f(n) = 2n$. Podemos, ter:
    \begin{itemize}
        \item $Y = \mathbb{N}$. $f$ é injetiva, mas não é sobrejetiva.
        \item $Y = \mathbb{R}$. $f$ é injetiva, mas não é sobrejetiva.
        \item $Y = \{n \in \mathbb{N} | n~par\}$. $f$ é bijetiva.
    \end{itemize}
\end{example}

\section{Funções e Subconjuntos do Domínio}

Nós podemos querer saber o comportamento de uma função em algum subconjunto
$A$ de  $X$. Por exemplo, podemos dizer que $f$ é injetiva em $A$ se para
todo $x_1$ e $x_2$ em A, $f(x_1) = f(x_2) $ implicar $x_1 = x_2 $.

\begin{definition}
    \label{def8}
    Se $f$ é função de $X$ e $Y$, dizemos que $f[A]$ denota a imagem de $f$ em
    $A$, definido por $f[A] = \{y \in Y | ~\exists x \in A, y = f(x)\}$.
\end{definition}

\begin{theorem}
    \label{prop7}
    Seja $f: X \to Y $ e $A$ um subconjunto de $X$. Então, para todo $x$ em
    $A$, $f(x)$ está em $f[A]$.
\end{theorem}

\begin{proof}
    Por definição, $f(x) \in f[A] $ se, e somente se, existe $x'$ em $A$ tal que
    $f(x') = f(x)$. Isto vale para  $x' = x $.
\end{proof}

\begin{theorem}
    \label{prop8}
    Seja $f: X \to Y$ e $g: Y \to Z$. Seja $A$ subconjunto de $X$. Então
    $$(g \circ f)[A] = g[f[A]]$$
\end{theorem}

\begin{proof}
    Seja $z \in (g \circ f)[A]$. Então, para algum $x \in A$, $z = (g \circ f)(x) = g(f(x))$.
    Pelo que acabamos de provar na Proposição \ref{prop8}, $f(x) \in f[A]$. Novamente, pelo
    que acabamos de provar, $g(f(x)) \in g[f[A]] $.

    Alternativamente, seja $z \in g[f[A]]$. Então, existe $y$ em $f[A]$ tal que
    $f(y) = z$. Como $y \in f[A]$, existe $x \in A$, tal que $f(x) = y$. Então
    $(g \circ f)(x) = g(f(x)) = g(y) = z$, então $z \in (g \circ f)[A] $
\end{proof}

Uma prova de que a composição de funções sobrejetivas é
sobrejetiva é a que descrevemos cima, pois $f: X \to Y$ é
sobrejetiva se, e somente se, $f[X] = Y$.

Nós podemos ver $f$ como uma função de $A$, um subconjunto de $X$ a $Y$, simplesmente
ignorando o comportamento de $f$ nos elementos fora de $A$.

\begin{definition}
    \label{def9}
    Denotamos $f \upharpoonright A$ como restrição de $f$ para $A$. Isto é, dadas $f: X \to Y$
    e $A \subseteq X$, $f \upharpoonright A : A \to Y$ é definida por $(f \upharpoonright A)(x) = x$,
    para todo $x$ em $A$.
\end{definition}

Agora, $f$ é injetiva em $A$ significa que a restrição de $f$ em $A$ é injetiva.

\begin{definition}[Pré-imagem]
    \label{def10}
    Se $f: X \to Y$ e $B \subseteq Y$, então a pré-imagem de $B$ em $f$, denotado por $f^{-1}[B]$ é
    definida por $f^{-1}[B] = \{x \in X | f(x) \in B\}$. Ou seja, é o conjunto de elementos de $X$ que
    são mapeados em $B$.
\end{definition}

Note que essa definição faz sentido mesmo que $f$ não tenha inversa, visto que dado $y \in B$ pode não
haver $x \in X$ com a propriedade de $f(x) \in B$, como podem ter vários. Se $f$ tem inversa ($f^{-1}$),
então, para todo $y$ em $B$, existe exatamente um elemento $x$ em $X$ com $f(x) \in B$. Neste caso,
dizemos que $f^{-1}[B]$ é a imagem de $B$ sobre $f^{-1}$ ou a pré-imagem de $B$ sobre $f$.

\begin{theorem}
    Seja $f: X \to Y$, $g: Y \to Z$ e $C \subseteq Z$. Então $$(g \circ f)^{-1}[C] = f^{-1}[g^{-1}[C]]$$.
\end{theorem}
\begin{proof}
    Para qualquer $y$, $y \in (g \circ f)^{-1}[C]$ se, e somente se, $g(f(y))$ está em $C$.
    Isto acontece se, e somente se, $f(y) \in g^{-1}[C]$, o que acontece se, e somente se $y \in f^{-1}[g^{-1}[C]]$.
\end{proof}

Seguem algumas propriedades sobre imagens e pré-imagens. Aqui, $f$ denota uma função
arbitrária de $X$ em $Y$. $A, A_1, A_2, ...$ denotam subconjuntos arbitrários de $X$,
e $B, B_1, B_2,...$ denotam subconjuntos arbitrários de $Y$.

\begin{enumerate}
    \item $A \subseteq f^{-1}[f[A]]$, e se $f$ é injetiva, $A = f^{-1}[f[A]]$.
    \item $f[f^{-1}[B]] \subseteq B$, e se $f$ é sobrejetiva, $B = f[f^{-1}[B]]$.
    \item Se $A_1 \subseteq A_2$, então $f[A_1] \subseteq f[A_2]$.
    \item Se $B_1 \subseteq B_2$, então $f^{-1}[B_1] \subseteq f^{-1}[B_2]$.
    \item $f[A_1 \cup A_2] = f[A_1] \cup f[A_2]$.
    \item $f^{-1}[B_1 \cup B_2] = f^{-1}[B_1] \cup f^{-1}[B_2]$.
    \item $f[A_1 \cap A_2] \subseteq f[A_1] \cap f[A_2]$ e, se $f$ é injetiva, $f[A_1 \cap A_2] = f[A_1] \cap f[A_2]$.
    \item $f^{-1}[B_1 \cap B_2] = f^{-1}[B_1] \cap f^{-1}[B_2]$.
    \item $f[A] \setminus f[B] \subseteq f[A\setminus B]$.
    \item $f^{-1}[A] \setminus f^{-1}[B] \subseteq f[A \setminus B]$.
    \item $f[A] \cap B = f[A \cap f^{-1}[B]]$.
    \item $f[A \cup f^{-1}[B]] \subset f[A] \cup B $.
    \item $A \cap f^{-1}[B] \subseteq f^{-1}[f[A] \cap B]$.
    \item $A \cup f^{-1}[B] \subseteq f^{-1}[f[A] \cup B]$.
\end{enumerate}

A partir de agora, vamos demonstrar algumas dessas proporições, ora
com descrições da demonstração e ora com provas em Lean. Tambémm sugerimos
que essas proposições sejam exercícios para a leitora ou o leitor.

\subsection{Demonstrações com linguagem natural}

\begin{theorem}[Item 7]
    \label{exerc1}
    Sejam $X$ e $Y$ conjuntos, $A_1, A_2 \in X$ e $f: X \to Y$.
\end{theorem}

\begin{proof}
    Se $y \in f[A_1 \cap A_2]$, temos que existe $x \in A_1 \cap A_2$, com $f(x) = y$.
    Nesse caso, $f(x) \in f[A_1]$, e $f(x) \in f[A_2]$, o que demonstra a primeira parte.

    Agora, suponha a injetividade de $f$. Suponha também que $y \in f[A_1] \cap f[A_2]$. Assim, existem
    $x_1 \in A_1$ e $x_2 \in A_2$, com as propriedades de $f(x_1) = y = f(x_2)$. Como a função é injetiva,
    $f(x_1) = f(x_2) \Rightarrow x_1 = x_2$. Assim $x_1 \in A_2$, que implica $x_1 \in A_1 \cap A_2$. Logo,
    $y \in f[A_1 \cap A_2]$.
\end{proof}

\begin{theorem}[Item 11]
    \label{exerc2}
    Sejam $X$ e $Y$ conjuntos, $f: X \to Y, A \subseteq X$ e $B \subseteq Y$. Então,
    $f[A] \cap B = f[A \cap f^{-1}[B]]$
\end{theorem}

\begin{proof}
    Suponha, inicialmente, que $y \in f[A] \cap B$. Então $y \in B$ e para algum $x \in A$, $f(x) = y$.
    Nesse sentido, $x \in f^{-1}[B]$. Assim, $x \in A \cap f^{-1}[B]$ e, portanto, $y \in f[A \cap f^{-1}[B]]$.

    Alternativamente, se $y \in f[A \cap f^{-1}[B]]$, existe $x \in A \cap f^{-1}[B]$, com $f(x) = y$.
    Daqui, concluímos que $y = f(x) \in f[A]$ e $y \in B$, pela definição de pré-imagem. Então $y \in f[A] \cap B$,
    como queríamos provar.
\end{proof}

\subsection{Demontrações em Lean}

Nós podemos utilizar variáveis limitadas para falar sobre o comportamento de funções
em conjuntos particulares.

\begin{lstlisting}
import data.set -- inclui o símbolo de subconjunto da imagem ''
open set function

variables {X Y : Type}
variables (A  : set X) (B : set Y)

def maps_to (f : X → Y) (A : set X) (B : set Y) :=
    ∀ x ∈ A, f x ∈ B

def inj_on (f : X → Y) (A : set X) :=
    ∀ (x₁ ∈ A) (x₂ ∈ A), f x₁ = f x₂ → x₁ = x₂

def surj_on (f : X → Y) (A : set X) (B : set Y) :=
    B ⊆ f '' A

\end{lstlisting}

A definição de \lstinline{maps_on} é a ideia de que a imagem de um subconjunto do
domínio $X$ está totamente inclusa em um subconjunto específico do contradomínio. As
definições de \lstinline{inj_on} e \lstinline{surj_on} são as definições usuais.

\begin{theorem}[Item 3]
\end{theorem}

\begin{theorem}[Item 9]
\end{theorem}

Observe que para o item 9, trataremos $X$ e $Y$ como conjuntos iguais. Caso não sejam,
$f[B]$ pode nem estar definida.

\begin{lstlisting}
import data.set
open set function

variables {X Y : Type}
variables (A B A₁ A₂ : set X)

theorem item3 (f: X → Y) : A₁ ⊆ A₂ → f '' A₁ ⊆ f '' A₂ :=
    assume h : A₁ ⊆ A₂,
    assume y,
    assume h₁ : y ∈ f '' A₁,
    have h₂ : ∃ x, x ∈ A₁ ∧ f(x) = y, from h₁,
    show y ∈ f '' A₂, from exists.elim h₂
        (assume (x' : X) (ha: x' ∈ A₁ ∧ f(x') = y ),
        have h₃ : x' ∈ A₂ ∧ f(x') = y, from and.intro (h ha.left)  ha.right,
        show y ∈ f '' A₂, from exists.intro x' h₃)

theorem item9 (f: X → X) : f '' A \ f '' B ⊆ f '' (A \ B) :=
begin
    intros y h,
    have h₁ : y ∈ f '' A, from mem_of_mem_diff h,
    have h₂ : ¬ (y ∈ f '' B), from not_mem_of_mem_diff h,
    apply exists.elim h₁,
    intros x h₃,
    apply exists.intro x,
    apply and.intro,
    apply mem_diff_of_mem,
        exact h₃.left,
        assume h₄ : x ∈ B,
        exact false.elim (h₂ (exists.intro x (and.intro h₄ h₃.right))),
        exact h₃.right
end

#check @mem_of_mem_diff
#check @not_mem_of_mem_diff
#check @mem_diff_of_mem

\end{lstlisting}

As funções \lstinline{mem_of_mem_diff, not_mem_of_mem_diff} e \lstinline{mem_diff_of_mem}
tem o objetivo de lidar com a diferença esua relação com a interseção. Note que usar táticas
auxilia o passo a passo, porém dificulta o posterior entendimento. Por isso, é recomendável,
nesses exemplos, utilizar alguma ferramenta para Lean. 
Essas funções e outras já apresentadas, encontram-se na biblioteca do Lean e grande parte delas está 
disponível quando você abre o namespace \lstinline{function}.

\begin{lstlisting}
open function 

#check @comp 
#check @has_left_inverse
-- A right_inverse tem duas definicoes para Lean. Apenas uma
-- esta no namespace function. Por isso e importante especificar. 
#check @function.right_inverse    
\end{lstlisting}

\section{Lógica de Segunda ou Mais Alta Ordem}

\subsection{Definindo a inversa Classicamente}

Para definir funções inversas, é necessário que utilizemos o racionínio clássico. 

\begin{lstlisting}
open classical 

#check @axiom_of_choice
#check @some_spec

variables A B : Type
variable P : A → Prop   
variable R : A → B → Prop

example : (∀ x, ∃ y, R x y) → ∃ f : A → B, ∀ x, R x (f x) :=
axiom_of_choice

example (h : ∃ x, P x) : P (some h) :=
some_spec h    
\end{lstlisting}

O axioma da escolha fala que se para todo \lstinline{x : X}, existe \lstinline{y: Y} com 
\lstinline{R x y}, então existe uma função \lstinline{f: X → Y} que para todo \lstinline{x},
escolhe-se \lstinline{y}. Em Lean, é utilizada a função \lstinline{some} para mostrar este a 
axioma, através da construção clássica. Podemos, portanto, definir a inversa como: 

\begin{lstlisting}
open classical function
local attribute [instance] prop_decidable

variables {X Y : Type}

noncomputable def inverse (f : X → Y) (default : X) : Y → X :=
λ y, if h : ∃ x, f x = y then some h else default   
\end{lstlisting}

Observe que a definição é não computacional, visto que como já argumentamos, para uma determinada função, essa 
hipótese \lstinline{h} pode não ser possível de validar algoritmicamente e, se a hipótese for válida, pode não 
ser possível encontrar um valor de \lstinline{x} adequado, também algoritmicamente. Também observe que essa inversa é 
definida assumindo \lstinline{y}, logo ela é definida para todo valor que ele assume e retorna algum valor qye cumpre 
essa propriedade \lstinline{f x = y}, quando a hipótese é válida, ou um valor padrão dado inicialmente, caso não exista. 
O comando \lstinline{local attribute} fala para a instância que a regra \lstinline{prop_decidable} vai ser utilizada no 
arquivo que se segue. Nesse caso, importamos os axiomas clássicos e tornamos disponível a instância genérica da decição.   

Podemos, então, demonstrar os seguintes teoremas. 

\begin{lstlisting}
open classical function
local attribute [instance] prop_decidable

variables {X Y : Type}

noncomputable def inverse (f : X → Y) (default : X) : Y → X :=
λ y, if h : ∃ x, f x = y then some h else default

theorem inverse_of_exists (f : X → Y) (default : X) (y : Y)
    (h : ∃ x, f x = y) : f (inverse f default y) = y :=
    have h₁ : inverse f default y = some h, from dif_pos h,
    have h₂ : f (some h) = y, from some_spec h,
eq.subst (eq.symm h₁) h₂

#check @dite
#check @dif_pos 

-- Using Term Mode 
theorem is_left_inverse_of_injective (f : X → Y) (default : X)
    (injf : injective f) : left_inverse (inverse f default) f :=
let finv := (inverse f default) in
    assume x,
    have h1 : ∃ x', f x' = f x, from exists.intro x rfl,
    have h2 : f (finv (f x)) = f x, 
        from inverse_of_exists f default (f x) h1,
    show finv (f x) = x, from injf h2

-- Using Tatic Mode 
theorem is_left_inverse_of_injective2 (f : X → Y) (default : X)
    (injf : injective f) : left_inverse (inverse f default) f :=
begin 
    intro x,
    apply injf, 
    apply inverse_of_exists,
    apply exists.intro x,
    exact eq.refl (f x), 
end
\end{lstlisting}

Observado o significado de \lstinline{dite}, que expressa a validade de um 
\lstinline{α : Sort}, independente de \lstinline{c: Prop} e o significado 
de \lstinline{dif_pos}, que expressa a igualdade entre uma expressão do formato 
\lstinline{dite} e outra do mesmo \lstinline{Sort}, podemos entender o siginificado
dessa prova. A versão em Táticas foi inserida como instrução de aplicação em comparação
como o modo em termos. 

\section{Funções e Relações}
