\chapter{Funções}

Desde o final do século XIX, diversas áreas da matemática 
estudam consistentemente \textit{\hyperlink{chapter.4}{conjuntos}}, 
\textit{\hyperlink{chapter.5}{relações}}, já apresentadas 
nesse texto, e funções. Neste capítulo, debruçaremos nossa 
atenção nas propriedades dessa terceira área. 

Entende-se que uma função $f$ é um mapeamento entre um 
domínio $X$ e um domínio $Y$, este que será conhecido como 
contradomínio, posteriormente. Entretanto, para teóricos 
de conjuntos, esses domínios são simplesmente considerados 
conjuntos. Vimos que \textit{\hyperlink{chapter.2}{Lean}} é uma 
linguagem baseada em Tipos. Dessa forma, estabelece-se uma 
diferença clara entre o Domínio $X$, caracterizado pelos Tipos, e
o Conjunto $A$, que é do tipo (sub)conjunto de $X$. Em Lean: 
\begin{lstlisting}
    variable X : Type
    variable A : set X    
\end{lstlisting}

Entretanto a visão da função como mapeamento entre conjuntos
é comum entre os matemáticos e será considerada nesse texto,
fazendo as devidas comparações com a linguagem de referência, 
o Lean. 

\section{O Conceito de Função}

Considere dois conjuntos quaisquer $X$ e $Y$ e um mapeamento $f$ 
do conjunto $X$ para o conjunto $Y$. Se $f$ atribui um e apenas 
um valor para cada elemento de $X$, dizemos que $f$ é uma função total, e
escrevemos $f: X \to Y$.  Chamamos $X$ de domínio de $f$, enquanto 
$Y$ é o contradomínio de $f$ e $\forall x, (x \in X  \Rightarrow f(x) \in Y)$. 
Nesse sentido, $\forall x_1 \in X, \forall x_2 \in X, x_1 = x_2 \Rightarrow f(x_1) = f(x_2)$.
Uma função também pode ser parcial quando ela não é definida para alguns valores 
do domínio. Como toda função parcial é total ao restringir o domínio, trateremos 
nesse texto das funções totais. 

A maneira mais simples de se representar uma função é escrevê-la explicitamente 
para cada elemento do domínio. Por exemplo, podemos escrever as seguintes expressões:

\begin{itemize}
    \item Seja $f: \mathbb{N} \to \mathbb{N}$ definida por 
    $f(n) = 2\cdot n + 1$
    \item Seja $g : \mathbb{N} \to \mathbb{R}$ definida por 
    $g(n) = \frac{n}{n+1}$
    \item Seja $h : \mathbb{R} \to \{0,1\}$ definida por 
    $$\left \{ \begin{array}{c}
    h(x) = 1 ~if~x \in \mathbb{Q} \\
    h(x) = 0 ~ if ~ x \not \in \mathbb{Q} \\
    \end{array}
    \right. $$
 \end{itemize}

A questão que se levanta é: o que torna uma expressão explícita legítima? 
Neste momento, deixaremos essa questão de lado e notaremos que a matemática 
é confortável com diversos tipos de definições, como, por exemplo, a definição 
de $h(x)$ no exemplo acima. Nesse mesmo exemplo, não fica claro em como podemos 
definir algoritmicamente se um número real de entrada é um número racional ou não.
Porém, esse não é o objetivo do capítulo.

Note que a escolha das variáveis $x$ e $n$ são arbitrárias. 
Isto nos leva à definição no capítulo de \textit{\hyperlink{chapter.2}{FOL}} de 
variável ligada (\textit{bound}), visto que se renomearmos $x$ com $y$, os valores
continuam os mesmos. 

Lógicos frequentemente utilizam a notação $\lambda x e(x)$ para denotar a função
que mapeia $x$ para $e(x)$. Essa notação chama-se \textit{notação lambda} e pode 
ser usada da seguinte forma: $f = \lambda x(x + 1)$, que significa $f(x) = x + 1$. 
Essa notação é mais interessante para cientistas da computação e lógicos do que 
propriamente para matemáticos. Em Lean, definimos da seguinte forma: 

\begin{lstlisting}
    variables X Y : Type
    variable f : X → Y    
\end{lstlisting}

Lembre-se que diferenciamos o tipo $X$ do tipo $set~X$ em Lean.

\section{Primeiras Definições}

\theoremstyle{definition}
\newtheorem{definition}{Definição}[section]

\theoremstyle{definition}
\newtheorem{example}{Exemplo}[section]

\theoremstyle{plain}
\newtheorem{theorem}{Proposição}[section]

\theoremstyle{plain}
\newtheorem{corollary}{Corolário}[section]

\begin{definition}
    \label{def1}
    Seja um conjunto $X$. A função identidade de $A$ é a função 
    $i_A : A \rightarrow A$ definida para todos os valores $x \in A$ tal que $i_A(x) = x$.
\end{definition}

\begin{definition}
    \label{def2}
    Sejam $f : X \rightarrow Y$ e $g : Y \rightarrow Z$ funções. 
    Defina $k : X \rightarrow Z$ por $k(x) = g(f(x))$. A função $k$ é chamada de composição
    de $f$ e $g$ ou $f$ composta com $g$ e é escrita $g \circ f$. Desta forma, para cada elemento, 
    primeiro  avaliamos $f(x) \in Y$ e depois avaliamos $g(f(x)) \in Z$. 
\end{definition}

Podemos ver em Lean as definições de composição e de identidade. Note que estamos em um namespace 
hidden para que não haja conflito de definições. 

\begin{lstlisting}
    namespace hidden
        variables {X Y Z : Type}

        def comp (f : Y → Z) (g : X → Y) : X → Z :=
        λx, f (g x)

        infixr  ` ∘ ` := comp

        #check @comp

        def id (x : X) : X := x

        #check @id
        
    end hidden
\end{lstlisting}

\begin{definition}
    \label{def3}
    Considere $f,g : X \to Y$. Dizemos que essas funções são iguais, 
    quando para todos os valores do domínio $X$, a correspondência no contradomínio $Y$ é a 
    mesma. Em lógica simbólica, $\forall x (x \in X \to (f(x) = g(x)) \iff f = g $. 
\end{definition}

Obseve que em termos de lógica formal de tipos, poderíamos reescrever como 
$\forall x : X (f(x) = g(x)) \iff f = g$. Escrevemos essa equivalência entre lógica e funções 
utilizando a extensionalidade das funções, muito semelhante à descrita no capítulo de 
\textit{\hyperlink{chapter.5}{conjuntos}}. Por exemplo, se $f, g : \mathbb{R} \to \mathbb{R}$ 
definidas por $f(x) = x + 1$ e $g(x) = 1 + x$, então $f = g$, pois para cada valor de $x$, vale 
a comutativade da soma. 

Em lean, o comando \lstinline{funext} (de "function extensionality") prova a igualdade de funções.

\begin{lstlisting}
    variables {X Y : Type}

    example (f g : X → Y) (h : ∀ x, f x = g x) : f = g :=
        funext h
\end{lstlisting}

\begin{theorem}
    \label{prop1}
    Para todo $f: X \to Y$, $f \circ i_X = f$ e $i_Y \circ f = f$
\end{theorem}
\begin{proof}
    Seja $x$ um elemento qualquer de $X$. Então $(f \circ i_X)(x) = f(i_X(x)) = f(x)$
     e $(i_Y \circ f)(x) = i_Y(f(x)) = f(x)$, o que mostra a igualdade. 
\end{proof}

No Lean, podemos mostrar essa proposição da seguinte forma. Para algumas funções, será necessário
escrevermos \lstinline{open function}.   

\begin{lstlisting}
    variables {X Y Z W : Type}

    lemma left_id (f : X → Y) : id ∘ f = f := rfl

    lemma right_id (f : X → Y) : f ∘ id = f := rfl

    theorem comp.assoc (f : Z → W) (g : Y → Z) (h : X → Y) :
        (f ∘ g) ∘ h = f ∘ (g ∘ h) := rfl
  
\end{lstlisting}

\begin{definition}
    \label{def4}
    Suponha $f:X \to Y$ e $g : Y \to X$ satisfaz $g \circ f = i_X$. Assim, 
    $g(f(x)) = i_X(x) = x$, para todos os elementos de $X$. Neste caso $g$ é dita inversa à esquerda
    de $f$ e $f$ é dita inversa à direita de $g$. Quando $g$ é inversa à direita e é inversa à esquerda
    de $f$, então $g$ é dita simplesmente inversa de $f$. 
\end{definition}

\begin{example}
    \label{ex1}
    Defina $f,g : \mathbb{R} \to \mathbb{R}$ por $f(x) = x + 1$ e $g(x) = x - 1$. Então $g$ 
    é inversa à direita e à esquerda de $f$ e vice-versa.
\end{example}
\begin{example}
    \label{ex2}
    $\mathbb{R}^{+}$ denota os reais não negativos. Defina $f : \mathbb{R} \to \mathbb{R}^{+}$ 
    por $f(x) = x^2$ e defina $g : \mathbb{R}^2 \to \mathbb{R}$ por $g(x) = \sqrt{x}$. Então 
    $f(g(x)) = (\sqrt{x})^2 = x$, para todo $x$ no domínio de $g$. Então $f$ é inversa à esquerda de $g$
    e $g$ inversa à direita de $f$. Por outro lado, $g(f(x)) = \sqrt{x^2} = |x|$. Logo $g$ não é inversa à 
    esquerda de $f$ e $f$ não é inversa à direita de $g$. 
\end{example}

\begin{theorem}
    \label{prop2}
    Suponha que $f : X \to Y $ tem inversa à esquerda, $h : Y \to X $ e uma inversa à direita, $k : Y \to X $.
    Então $h = k$. 
\end{theorem}
\begin{proof}
    Seja $y \in Y$. $h(f(k(y))) = k(y)$, pois $h$ é uma inversa à esquerda de $f$. 
    Por outro lado, $f(k(y)) = y$ e, portanto $h(f(k(y))) = h(y)$. Assim $k(y) = h(y) $ e as funções são iguais. 
\end{proof}

Essa proposição pode também ser vista em Lean. Considerarei a prova utilizando táticas. O leitor pode estudar 
aquela que sentir mais confortável. Note que neste exemplo, já foi necessário o \lstinline{namespace function}, 
visto que \lstinline{left_inverse} e \lstinline{right_inverse} já estão predefinidas. 

\begin{lstlisting}
    open function 

    variables {X Y Z: Type}

    -- Term and Calc Mode
    example (f: X → Y) (h: Y → X) (k: Y → X) 
        (hf: left_inverse h f) (fk: right_inverse k f): h = k :=
        have H: ∀ ( x : Y ), h x = k x, from 
            assume x, 
            have h1: h (f (k x)) = k x , from calc
                h ( f (k x)) = k x: by apply hf, 
            have h2: h (f (k x)) = h x, from calc
                h (f (k x)) = h x: by rw fk,         
            show h x = k x, from eq.trans (eq.symm h2) h1,
        show h = k, from funext H

    -- Tatics Mode
    example (f: X → Y) (h: Y → X) (k: Y → X) 
        (hf: left_inverse h f) (fk: right_inverse k f): h = k :=
        begin
            apply funext, 
            assume x, 
            have hf: h (f (k x)) = k x, by apply hf, 
            rw ←hf,
            rw fk, 
        end
\end{lstlisting}

\begin{theorem}
    \label{prop3}
    Seja $f: X \to Y$. Se a inversa de $f$ existe, então ela é única. Isto é, se $g_1, g_2: Y \to X$ são inversas 
    de $f$, então $g_1 = g_2$        
\end{theorem}   
\begin{proof}
    Sabemos que $g_1$ é inversa à esquerda de $f$. Então, 
    $$g_1(f(g_2(x))) = g_2(x), ~para~todos~os~valores~de~x.$$ 
    Também, $g_2$ é inversa de $f$. Então, 
    $$g_1(f(g_2(x))) = g_1(x), ~para~todos~os~valores~de~x.$$ Logo $g_1 = g_2$. 
\end{proof}

Quando a inversa de $f$ existe, então, podemos escrevê-la como $f^{-1}$. Dada a Definição \ref{def4}, 
podemos afirmar que $(f^{-1})^{-1} = f$.  

Observe que uma função pode possuir mais de uma função inversa à esquerda ou mais de uma função inversa 
à direita. Ainda, quando uma função possui mais de uma função inversa à esquerda, ela não possui inversa 
à direita. Se ela possuisse, ela seria igual a todas as funções inversas à esquerda pela Proposição \ref{prop2}, 
portanto elas seriam iguais, o que é uma contradição, por hipótese. O mesmo vale para inversas à direita. 

\begin{theorem}
     \label{prop4}
     Seja $f: X \to Y $ e $g: Y \to Z$. Se $h: Y \to X$ e $k: Z \to Y$ são inversas à esqueda de $f$ e $g$, 
     respectivamente, então $h \circ k$ é inversa à esquerda de $g \circ f$. O mesmo vale quando substituimos
     esquerda por direita na proposição. 
\end{theorem}

\begin{proof}
 $(h \circ k)\circ(g \circ f)(x) = h(k(g(f(x)))) = h(f(x)) = x $. A demonstração quando substituimos a proposição
 de esquerda para direita é análoga e é deixada como exercício ao leitor.
\end{proof}

\begin{corollary}
    \label{cor1}
    Se $f: X \to Y$ e $g: Y \to Z$ possuem inversas, então $(f \circ g)^{-1}$ existe e 
    $(f \circ g)^{-1} = f^{-1} \circ g^{-1}$.
\end{corollary}

\begin{proof}
    Segue diretamente da Proposição \ref{prop4} e Proposição \ref{prop2}.  
\end{proof}

\section{Funções Injetiva, Sobrejetiva e Bijetiva}

\begin{definition}[Função Injetiva]
    \label{def5}
    Também conhecida como função injetora, é uma função em que elementos distintos do domínio são 
    mapeados para elementos diferentes do contradomínio. Dessa forma, seja $f: X \to Y$. Se 
    $\forall x_1 \in X, \forall x_2 \in X, x_1 \neq x_2 \Rightarrow f(x_1) \neq f(x_2) $, f é injetiva. 
    A definição pode ser feita pela contrapositiva dessa afirmação, também. A definição pela contrapositiva
    será bastante utilizada nas demonstrações. 
\end{definition}

\begin{definition}[Função Sobrejetiva]
    \label{def6}
    Também conhecida como função bijetora, é uma função em que todos os elementos do contradomínio 
    estão na imagem da função. Dessa forma, seja $f: X \to Y$. Se $\forall y \in Y, \exists x \in X, f(x) = y$, 
    f é sobrejetiva. 
   \end{definition}

\begin{definition}[Função Bijetiva]
    \label{def7}
    Também conhecida como bijeção ou correspondência um a um, é uma função simultaneamente injetiva e 
    sobrejetiva. 
\end{definition}

Em Lean, essas definições são descritas no \lstinline{namespace function}. 

\begin{example}
    Considere $X = {1,2,3,4}$ e $Y = \{John, Paul, George, Ringo\}$. É possível construir uma função bijetiva 
    entre o conjunto $X$ e o conjunto $Y$.  
\end{example}

\begin{lstlisting}
    variables {X Y: Type}

    def injective (f : X → Y) : Prop :=
    ∀ x₁ x₂, f x₁ = f x₂ → x₁ = x₂

    def surjective (f : X → Y) : Prop :=
    ∀ y, ∃ x, f x = y

    def bijective (f : X → Y) := injective f ∧ surjective f
\end{lstlisting}

\begin{theorem}
    \label{prop5}
    Seja $f : X \to Y$.
    \renewcommand{\labelenumi}{\Roman{enumi}}
    \begin{enumerate}
        \item Se $f$ possui inversa à esquerda, $f$ é injetiva.
        \item Se $f$ possui inversa à direita, $f$ é sobrejetiva. 
        \item Se $f$ possui inversa, $f$ é bijetiva. 
    \end{enumerate}
\end{theorem}
\begin{proof}
    Para provar I), suponha que $f(x_1) = f(x_2)$ e que $g$ é inversa à esquerda
    de $f$. Assim $g(f(x_1)) = x_1$ e $g(f(x_2)) = x_2)$. Portanto $x_1 = x_2$ e 
    está provado. Para provar II), considere $h$ inversa à direita de $f$. Seja 
    $y \in Y$ e $x = h(y)$. Então $f(x) = f(h(y)) = y$. A terceira sai diretemente 
    das duas anteriores.  
\end{proof}

\begin{lstlisting}
    open function 

    variables {X Y : Type}

    theorem inj_of_left_inverse {g : Y → X} {f : X → Y} :
    left_inverse g f → injective f :=
        assume h, assume x₁ x₂, assume feq, 
        calc x₁ = g (f x₁) : by rw h
            ... = g (f x₂) : by rw feq
            ... = x₂       : by rw h

    theorem surj_of_right_inverse {g : Y → X} {f : X → Y} :
    right_inverse g f → surjective f :=
        assume h, assume y,
        let  x : X := g y in
        have f x = y, from calc
            f x  = (f (g y))    : rfl
            ... = y            : by rw [h y],
        show ∃ x, f x = y, from exists.intro x this
\end{lstlisting}

\begin{theorem}
    \label{prop6}
    Seja $f: X \to Y$
    \renewcommand{\labelenumi}{\Roman{enumi}}
    \begin{enumerate}
        \item Se $X$ é não vazio e $f$ é injetiva, então $f$ possui inversa à esquerda.
        \item Se $f$ é sobrejetiva, então $f$ possui inversa à direita.
        \item Se $X$ é não vazio e $f$ é bijetiva, então $f$ possui inversa.
    \end{enumerate}
\end{theorem}

\begin{proof}
    Seja $\hat{x} \in X$. Defina uma função $g: Y \to X$ com $g(y) = x$, tal que $f(x) = y$,
    caso exista esse $x$. Caso ele não exista, $g(y) = \hat{x}$. Agora, suponha que 
    $g(f(x)) = x'$. Pela definição de $g$, $f(x) = y$, logo $g(y) = x$. Assim, $f(x) = f(x')$.
    Como $f$ é injetiva $x = x'$. 

    Para a segunda afirmação, defina $h: Y \to X$, onde, para cada $y \in Y$, escolha um elemento 
    $x \in X$, tal que $h(y) = x$. Note que a sobrejetividade garante a existência desse elemento,
    mas não garante a unicidade. Então $f(h(y)) = f(x) = y$, por definição de $h$. 

    A fim de provar a terceira, basta as demonstrações das anteriores. 
\end{proof}

Algumas observações sobre essa demonstração são importantes. Ao definir $g$ na primeira parte, 
precisa-se decidir se $x \in X$ existe tal que $f(x) = y$. Isso pode não ser algoritmicamente 
feito, logo $g$ poderia não ser computável. Na construção de $h$, a prova requere que haja 
uma escolha de valor de $x$ entre os possíveis candidatos. Isto é uma versão do 
\href{https://pt.wikipedia.org/wiki/Axioma_da_escolha#Enunciado}{\textit{axioma da escolha}}. 
Este axioma foi muito debatido no século XX, mas hoje já é comum para demonstrações. O paradoxo
de Banach-Tarski é um argumento contra o axioma.

Utilizando conceitos e resultados da seção anterior, podemos provar a seguinte proposição.

\begin{theorem}
    Seja $f: X \to Y$ e $g: Y \to Z$. 
    \renewcommand{\labelenumi}{\Roman{enumi}}
    \begin{enumerate}
        \item Se $f$ e $g$ são injetivas, $g \circ f$ também será.
        \item Se $f$ e $g$ são sobrejetivas, $g \circ f$ também será.
    \end{enumerate}
\end{theorem}

\begin{proof}
    Basta aplicarmos a Proposição \ref{def4} e definirmos $h$ e $k$ como 
    inversas à esquerd ade $f$ e $g$ respefticamente. Logo $(h \circ k)$ 
    é inversa à esquerda de  $(g \circ f)$ e, portanto, ela é injetiva. 

    O mesmo vale para a segunda afirmação.
\end{proof} 

\begin{lstlisting}
    open function

    namespace hidden
        variables {X Y Z : Type}

        theorem injective_comp {g : Y → Z} {f : X → Y}
        (Hg : injective g) (Hf : injective f) :
        injective (g ∘ f) :=
            assume x₁ x₂,
            assume : (g ∘ f) x₁ = (g ∘ f) x₂,
            have f x₁ = f x₂, from Hg this,
            show x₁ = x₂, from Hf this

        theorem surjective_comp {g : Y → Z} {f : X → Y}
            (hg : surjective g) (hf : surjective f) :
        surjective (g ∘ f) :=
        begin
            assume z, 
            apply exists.elim (hg z),
            assume y (hy: g y = z),
            apply exists.elim (hf y),
            assume x (hx: f x = y),
            rw ←hx at hy,
            apply exists.intro x hy
        end

        theorem bijective_comp {g : Y → Z} {f : X → Y}
            (hg : bijective g) (hf : bijective f) : 
            bijective (g ∘ f) :=
        have ginj : injective g, from hg.left,
        have gsurj : surjective g, from hg.right, 
        have finj : injective f, from hf.left,
        have fsurj : surjective f, from hf.right,
        and.intro (injective_comp ginj finj) 
                  (surjective_comp gsurj fsurj)
    end hidden       
\end{lstlisting}

\begin{example}
    Considere $f: \mathbb{N} \to Y$, tal que $f(n) = 2n$. Podemos, ter:
    \begin{itemize}
        \item $Y = \mathbb{N}$. $f$ é injetiva, mas não é sobrejetiva. 
        \item $Y = \mathbb{R}$. $f$ é injetiva, mas não é sobrejetiva.
        \item $Y = \{n \in \mathbb{N} | n~par\}$. $f$ é bijetiva. 
    \end{itemize} 
\end{example}