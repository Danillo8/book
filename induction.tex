\chapter{Indução nos Números Naturais}

Aqui irá se escrever a introdução do capítulo . . . 

\section{Construção de Tipos por Indução}

Aqui será comentado a indução em geral em vários tipos, e depois têm uma deixa para a indução nos naturais . . .

\section{O Príncipio de Indução}

P(0) e P(n) -> P(n+1) e alguns exemplos . . .

\section{Variantes da Indução}

Indução começando em um natural k e indução completa . . .

\section{Recursão}

Algumas definições recursivas como Fibonacci, somatório, produtório, e depois alguns exemplos . . .

\section{Operadores Aritméticos}

Operadores aritméticos em um contexto global . . .

\subsection{Aritmética nos Naturais}

Operadores aritméticos nos naturais . . .

No Lean, os naturais são definidos internamente como um tipo indutivo:

\begin{lstlisting}
namespace hidden

inductive nat : Type
| zero : nat
| succ : nat → nat

end hidden
\end{lstlisting}

Como não queremos que o nosso novo tipo \lstinline{nat} entre em conflito com o \lstinline{nat} do Lean, usa-se o \lstinline{namespace}. Com ele, \lstinline{nat} é reconhecido como \lstinline{hidden.nat} e suas respectivas propriedades também, \lstinline{hidden.nat.zero} e \lstinline{hidden.nat.succ}. O símbolo unicode $\mathbb{N}$ pode ser escrito com \textit{\textbackslash N} ou \textit{\textbackslash nat}, e equivale à mesma coisa que \lstinline{nat}.

Apesar de termos definido o tipo \lstinline{nat} anteriormente, usaremos o da biblioteca do Lean chamada \lstinline{nat}. Como este tipo é definido indutivamente, podemos criar funções recursivas e provar teoremas por indução. Por exemplo:

\begin{lstlisting}
open nat

def two_pow : ℕ → ℕ
| 0        := 1
| (succ n) := 2 * two_pow n

def fact : ℕ → ℕ
| 0        := 1
| (succ n) := (succ n) * fact n
\end{lstlisting}

No trecho de código acima, definimos duas funções de maneira recursiva: $2^n$ e $x!$. Como o domínio são os naturais, as funções necessitam saber o que fazer com toda entrada possível, que no nosso caso seria a constante $0$ e um outro natural qualquer. Quando colocado \lstinline{(succ n)}, o Lean interpreta como se a entrada fosse \lstinline{(succ n)}, isto é, $n$ torna-se o valor tal que \lstinline{(succ n)} é igual à entrada. Com isso, podemos atribuir o valor desejado recursivamente à nossa função.

O infixo $+$ é criado de uma maneira que $n + 1$ é reconhecido da mesma maneira que \lstinline{(succ n)}. Todas definições acima funcionariam da mesma maneira caso este fosse substituído.

Caso se deseje criar uma função de potência mais genérica, o seguinte código funciona muito bem e inclusive é a definição  da biblioteca \lstinline{nat}, que pode ser chamada por \textit{nat.pow}.

\begin{lstlisting}
def pow : ℕ → ℕ → ℕ
| m 0        := 1
| m (n + 1)  := pow m n * m
\end{lstlisting}

Repare que a recursão ocorre no segundo parâmetro. Com esta nova função, podemos provar alguns teoremas:

\begin{lstlisting}
open nat

theorem pow_zero (n : ℕ) : pow n 0 = 1 := rfl
theorem pow_succ (m n : ℕ) : pow m (n+1) = pow m n * m := rfl
\end{lstlisting}

O operador \lstinline{rfl} funciona da seguinte maneira: ele tenta simplificar ambos os lados da igualdade seguindo definições e propriedades. Caso ele consiga chegar na igualdade desejada, ele não retorna um erro.

Da mesma maneira que o infixo $+$ foi definido, também é definido o infixo "\textasciicircum ". A fórmula \lstinline{m^n} equivale a escrever \lstinline{pow m n}. Repare que no teorema \lstinline{pow_succ} poderiamos ter invertido \lstinline{m^n * m} por \lstinline{m * m^n} e usado a comutatividade da multiplicação. Entretanto caso não se deseje usa-lá, podemos realizar uma prova em Lean por indução:

\begin{lstlisting}
open nat

theorem pow_succ' (m n : ℕ) : ∀ m n : nat, m^(succ n) = m * m^n :=
assume m n : nat,
nat.rec_on n 
  (show m^(succ 0) = m * m^0, from calc
    m^(succ 0) = m^0 * m : by rw pow_succ
           ... = 1 * m   : by rw pow_zero
           ... = m       : by rw one_mul
           ... = m * 1   : by rw mul_one
           ... = m * m^0 : by rw pow_zero)
  (assume n,
    assume ih : m^(succ n) = m * m^n,
    show m^(succ (succ n)) = m * m^(succ n), from calc
      m^(succ (succ n)) = m^(succ n) * m   : by rw pow_succ
                    ... = (m * m^n) * m    : by rw ih
                    ... = m * (m^n * m)    : by rw mul_assoc
                    ... = m * m^(succ n)   : by rw pow_succ)
\end{lstlisting}

O comando \lstinline{nat.rec_on} recebe como entrada três parâmetros: a variável que se deseja aplicar indução; uma prova de $P(0)$ e uma prova de $P(n)→P(n+1)$, onde $P$ seria o \lstinline{goal}. No exemplo acima, usando \lstinline{calc} e algumas funções já definidas na biblioteca \lstinline{nat} e no Lean, conseguimos provar sem usar a comutatividade da multiplicação. Repare que como usamos \lstinline{open nat}, não há necessidade de escrever \lstinline{nat.pow}, apenas \lstinline{pow}.

Uma boa dica para provas usando \lstinline{calc} é de introduzir o seu resultado desejado e ir trabalhando a equação atual de modo a chegar ao seu objetivo. Segue abaixo um exemplo da identidade \lstinline{m^(n + k) = m^n * m^k}.

\begin{lstlisting}
open nat

theorem pow_add (m n k : ℕ) : m^(n + k) = m^n * m^k :=
nat.rec_on k
  (show m^(n + 0) = m^n * m^0, from calc
    m^(n + 0) = m^n       : by rw add_zero
          ... = m^n * 1   : by rw mul_one
          ... = m^n * m^0 : by rw pow_zero)
  (assume k,
    assume ih : m^(n + k) = m^n * m^k,
    show m^(n + succ k) = m^n * m^(succ k), from calc
      m^(n + succ k) = m^(succ (n + k)) : by rw nat.add_succ
                ... = m^(n + k) * m    : by rw pow_succ
                ... = m^n * m^k * m    : by rw ih
                ... = m^n * (m^k * m)  : by rw mul_assoc
                ... = m^n * m^(succ k) : by rw pow_succ)
\end{lstlisting}

Dessa vez, nossa indução foi em $k$. Uma dica interessante presente no editor web do Lean é que você pode passar o mouse por cima do nome do teorema e ver o que ele afirma. Usando \lstinline{calc} para estruturar o objetivo e \lstinline{rewrite} para completar a justificativa de cada passo torna sua prova muito mais legível e elegante.
\subsection{Aritmética nos Inteiros}

Operadores aritméticos nos inteiros . . .

\section{Exercícios}

Alguns exercícios de indução . . .