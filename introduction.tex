\chapter{Introdução}
A linguagem matemática é, atualmente, caracterizada por seu rigor formal, e possibilidade de abstração sobre diferentes contextos. Somos capazes de descrever e avaliar situações impensáveis mesmo externas a realidade paupável, e obter respostas para problemas que seriam impossíveis sem o uso desse rigor. De fato, o surgimento da matemática formal é frequentemente creditado a Grécia, sec. VI a.c., embora haja registros retratando uso de conhecimento matemático no Egito muito antes disso. Tal fato é justificado uma vez que apenas na Grécia surge o questionamento de \textit{como sabemos} e não apenas \textit{o que sabemos}. A partir daí, as bases para desenvolver um conhecimento sólido poderiam ser iniciadas, partindo da filosofia ao método científico e formalismo matemático.

%Pensar em uma forma deu juntar as duas partes sem perder informação

Existem evidências do uso da matemática já por volta de 3000 a.c, onde povos antigos como os mesopotâmicos, egípcios e moradores de Ebla utilizavam a aritimética, álgebra e a geometria para taxação, comércio, trocas, astronomia e na formulação de calendários e marcação do tempo corrente. 

Os registros matemáticos (em texto) mais antigos vem da Mesopotâmia e do Egíto, e são datados aproximadamente entre 2000 a.c e 1800 a.c, mas muitos estudiosos consideram a Grécia Antiga como o berço da matemática, onde, por volta de 600 a.c, os Pitagóricos iniciaram o estudo da matemática como uma "disciplina demonstrativa" e introduziram o raciocínio dedutivo e o rigor matemático em provas formais.

%Colocar segunda parte histórica?

Aqui, a lógica entra exatamente para questionar \textit{como sabemos}, ou melhor, como podemos ter certeza de que concluimos coisas acertadas partindo de um contexto inicial. A formalização do processo de raciocínio garante que \textit{construimos um castelo de cartas adicionando cola as partes}.

Não devemos, no entanto, nos limitar a discutir lógica como ferramenta relacionada a teoremas ou metafísica. Esse ramo é presente no estudo da linguagem (lógica informal), processos industriais e computação (métodos formais), por exemplo. %Pesquisadores imaginavam que, quando fossem capazes de descrever o pensamento humano em termos lógicos, seriam tambem capazes de produzir uma máquina inteligente.

\section{Lógica formal e linguagem natural}
% uma referencia interssante
% http://www.pucrs.br/edipucrs/online/pesquisa/pesquisa/artigo11.html
A linguagem humana é conhecidamente um emaranhado gigantesco de simbolos, gestos, palavras, sentidos, interpretações. Essa é fruto de lapidação através de milênios, do que se iniciou com uma linguagem básica animalesca, chegando a complexidade atual em um processo evolutivo constante; assim ocorreu com a fala e escrita.

É impossível afirmar, portanto, que uma língua, digamos o português, é ineficiente na transmissão de significados, sendo fruto de evolução pura e constante. De fato, os linguistas se dedicam em grande parte a desvendar os mecanismos segundo os quais a lingua se molda e se torna tão eficiente, transmitindo significados profundos, dificeis até mesmo de serem explicados formalmente. Tal língua capta noções como: medo; incerteza; possibilidade; instinto, etc. naturalmente imprecisas.

Em contraparte, a linguagem lógica busca firmar o processo de pensamento sob regras bem definidas, que às vezes não captam as noções da linguagem natural em sua completude, no entanto, permitem discutir assertivas que seriam contraintuitivas para um ouvinte desinformado. Por exemplo, podemos afirmar que \textit{toda vaca voadora come gente}, e isso não fará o menor sentido a não ser que estabeleçamos essa como uma afirmativa lógica. Esse é um exemplo caricato que esclarece como a lógica nos permite discutir conclusões acertadas, que seriam pouca acessíveis apenas utilizando de um arcabouço de linguagem natural.

[Acho que aqui podemos introduzir para discutir um problema de lógica usando linguagem natural. Mostrar como ficaria extensivo e pouco claro "Procure argumentar uma solução para o problema: "]

Mais uma vez, é importante reforçar que a linguagem natural é um meio extremamente eficiente de transmissão de significaos. Há um ramo da lógica chamado lógica informal que lida com o modo como expressamos o raciocínio através da lingua: argumentação e falácias, por exemplo. É possível discutir se \textit{Um ser humano sem linguagem enlouqueceria}.

\section{Sobre Sistemas Dedutivos}
% Um pouco sobre sistemas, DN, Cálculo de Sequentes e Resolução, pex.
Para motivação, considere o seguinte problema: \textit{Você acorda em uma sala com duas portas. Uma dá para a liberdade, e a outra leva a morte, mas você não sabe qual é cada uma. Junto a cada porta há um guarda. Um dos guardas é sempre sincero, enquanto o outro é sempre mentirozo. Você tem direito a uma única pergunta para decidir qual porta tomar. O que você faz?}

Esse é claramente um problema envolvendo um raciocínio complexo, e, se tratando da sua vida em jogo, exige plena certeza da solução antes de uma resposta definitiva. De fato, seremos capazes de formalizar problemas desse tipo, e obter métodos de derivação para obter um prova para a resposta.

\section{Provadores, ATP e ITP}
Como computadores tem sido usados no processo de provas desse tipo.

\chapter{Introdução ao LEAN}
Capítulo sobre o provador utilizado no curso: o LEAN

\section{Teoria dos tipos}
Descrição superficial do teoria dos tipos para justificar o pardigma PAT.
Tambem justifica como o LEAN provas as coisas.

\section{Provas usando LEAN}
Exemplos (avancados) de provas utilizado LEAN.
