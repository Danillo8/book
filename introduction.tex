\chapter{Introdução}
%A linguagem matemática é caracterizada por seu rigor formal, e possibilidade de abstração sobre diferentes contextos.
%Somos capazes de descrever situações impensáveis, e obter respostas para problemas que seriam impossíveis sem o uso desse rigor.

Há evidências do uso da matemática já por volta de 3000 a.c, onde povos antigos como os mesopotâmicos, egípcios e moradores de Ebla utilizavam a aritimética, álgebra e a geometria para taxação, comércio, trocas, astronomia e na formulação de calendários e marcação do tempo corrente. Os registros textuais mais antigos desse tipo de prática vem da Mesopotâmia e do Egíto, e são datados aproximadamente entre 2000 a.c e 1800 a.c, mas muitos estudiosos consideram a Grécia Antiga como o berço da matemática, onde, por volta de 600 a.c, os Pitagóricos iniciaram o estudo da matemática como uma "disciplina demonstrativa" e introduziram o raciocínio dedutivo e o rigor matemático em provas formais.

Apenas quando surge o questionamento de \textit{como sabemos}, e não apenas \textit{o que sabemos} as bases para desenvolver um conhecimento sólido são devidamente definidas.
Aqui, a lógica formal entra exatamente ao questionar \textit{como sabemos}, ou melhor, como temos certeza do que concluimos partindo de um contexto inicial.
A formalização do processo de raciocínio garante que \textit{construimos um castelo sólido}.
%A filosofia, o método científico e o formalismo matemático que conhecemos se devem em grande parte a essas bases bem definidas.

%Não devemos, no entanto, nos limitar a discutir lógica como ferramenta relacionada a teoremas ou metafísica. Esse ramo é presente no estudo da linguagem (lógica informal), processos industriais e computação (métodos formais), por exemplo.

\section{História Recente}
Aristóteles, Frege, Gentzen, Russel... Problemas abertos na lógica?

\section{Lógica formal e linguagem natural}
% uma referencia interssante
% http://www.pucrs.br/edipucrs/online/pesquisa/pesquisa/artigo11.html
A linguagem humana é conhecidamente um emaranhado gigantesco de simbolos, gestos, palavras, sentidos, interpretações.
Essa é fruto de lapidação através de milênios, do que se iniciou com uma linguagem básica animalesca, chegando a complexidade atual em um processo evolutivo constante; assim ocorreu com a fala e escrita.

É impossível afirmar, portanto, que uma língua, digamos o português, é ineficiente na transmissão de significados, sendo fruto de evolução pura e constante.
De fato, os linguistas se dedicam em grande parte a desvendar os mecanismos segundo os quais a lingua se molda e se torna tão eficiente, transmitindo significados profundos, dificeis até mesmo de serem explicados formalmente.
Tal língua capta noções como: medo; incerteza; possibilidade; instinto, etc. naturalmente imprecisas.

Em contraparte, a linguagem lógica busca firmar o processo de pensamento sob regras bem definidas, que às vezes não captam as noções da linguagem natural em sua completude, no entanto, permitem discutir assertivas que seriam contraintuitivas para um ouvinte desinformado.
Por exemplo, podemos afirmar que \textit{toda vaca voadora come gente}, e isso não fará o menor sentido a não ser que estabeleçamos essa como uma afirmativa lógica.
Esse é um exemplo caricato que esclarece como a lógica nos permite discutir conclusões acertadas, que seriam pouca acessíveis apenas utilizando de um arcabouço de linguagem natural.

[Acho que aqui podemos introduzir para discutir um problema de lógica usando linguagem natural. Mostrar como ficaria extensivo e pouco claro "Procure argumentar uma solução para o problema: "]

Mais uma vez, é importante reforçar que a linguagem natural é um meio extremamente eficiente de transmissão de significaos.
Há um ramo da lógica chamado lógica informal que lida com o modo como expressamos o raciocínio através da lingua: argumentação e falácias, por exemplo.
Vale a discussão de se \textit{um ser humano sem linguagem enlouqueceria}.

\section{Sobre Sistemas Dedutivos}
% Um pouco sobre sistemas, DN, Cálculo de Sequentes e Resolução, pex.
Para motivação, considere o seguinte problema: \textit{Você acorda em uma sala com duas portas. Uma dá para a liberdade, e a outra leva a morte, mas você não sabe qual é cada uma.
Junto a cada porta há um guarda. Um dos guardas é sempre sincero, enquanto o outro é sempre mentirozo. Você tem direito a uma única pergunta para decidir qual porta tomar. O que você faz?}

Esse é claramente um problema envolvendo um raciocínio complexo, e, se tratando da sua vida em jogo, exige plena certeza da solução antes de uma resposta definitiva.
De fato, seremos capazes de formalizar problemas desse tipo, e obter métodos de derivação para obter um prova para a resposta.

[imcompleto]

\section{Provadores, ATP e ITP}
% referencia que bem diferencia ATPs e ITPs
% além de direcionar o que desejamos sobre provadores
% https://leanprover.github.io/theorem_proving_in_lean/introduction.html

Verificação formal envolve o uso de lógica e, mais recentemente, de aparato computacional na descrição de um contexto em termos matemáticos suficientemente precisos, e estabelecimento de assertivas sobre essas definições.

A partir disso, passamos utilizar teoremas matemáticos para verificar assertivas sobre os elementos abstraídos, que podem bem representar todo tipo de objeto de interesse.
De fato, somos capazes de descrever matematicamente \textit{softwares}, processos industriais, sistemas de gerenciamento de cargas, ou teoremas sobre conjuntos, por exemplo, utilizando dos mesmos aparatos.
Isso significa que na prática, não há distinção entre ferramentas provando teoremas, ou verificando assertivas sobre um objeto qualquer de interesse.
De fato, reduzimos os problemas de verificação formal a problemas de provas de teoremas e, mais relevante, podemos utilizar computadores nessa tarefa.

Esse tipo de discussão se tornou possível após os recentes avanços no campo de estudo da lógica.
Reduzimos regras de inferencia e derivação presentes na mior parte das provas a um conjunto pequeno de axiomas fundamentais.

A redução no conjunto de regras viabilizaria computadores implementando ferramentas para auxílio a prova de teoremas, divididas em basicamente duas classes: auxílio a obtebção de uma prova, ou verificação de uma assertiva dada. Esses são os \textit{ATPs} e o \textit{ITPs}.

\subsection{Automated theorem proving}

falamos sobre os ATPs, alguns exemplos... Descrevemos características.

\subsection{Interactive theorem proving}

Damos características desse paradigma, exemplos...

%\subsection{O que é Automacão}
%fiquei interessafo em defender o que discutimos tantas vezes com o Rdmkr, sobre até onde, e como, o computador está automatizando as tarefas...
