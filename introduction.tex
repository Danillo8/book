\chapter{Introdução}

A linguagem matemática é, atualmente, caracterizada por seu rigor formal, e possibilidade de abstração sobre diferentes contextos. Somos capazes de descrever e avaliar situações impensáveis mesmo externas a realidade paupável, e obter respostas para problemas que seriam impossíveis sem o uso desse rigor. De fato, o surgimento da matemática formal é frequentemente creditado a Grécia, sec. VI a.c., embora haja registros retratando uso de conhecimento matemático no Egito muito antes disso. Tal fato é justificado uma vez que apenas na Grécia surge o questionamento de \textit{como sabemos} e não apenas \textit{o que sabemos}. A partir daí, as bases para desenvolver um conhecimento sólido poderiam ser iniciadas, partindo da filosofia ao método científico e formalismo matemático.

Aqui, a lógica entra exatamente para questionar \textit{como sabemos}, ou melhor, como podemos ter certeza de que concluimos coisas acertadas partindo de um contexto inicial. A formalização do processo de raciocínio garante que \textit{construimos um castelo de cartas adicionando cola as partes}.

Não devemos, no entanto, nos limitar a discutir lógica como ferramenta relacionada a teoremas ou metafísica. Esse ramo é presente no estudo da linguagem (lógica informal), processos industriais e computação (métodos formais), por exemplo. %Pesquisadores imaginavam que, quando fossem capazes de descrever o pensamento humano em termos lógicos, seriam tambem capazes de produzir uma máquina inteligente.

\section{Lógica formal e linguagem natural}
Para motivação, considre o seguinte problema: \textit{Você acorda em uma sala com duas portas. Uma dá para a liberdade, e a outra leva a morte, mas você não sabe qual é cada uma. Junto a cada porta há um guarda. Um dos gusrdas é sempre sincero, enquanto o outro é sempre mentirozo. Você tem direito a uma pergunta. O que você faz?}

\section{Sobre Sistemas Dedutivos}
Um pouco sobre sistemas, DN, Cálculo de Sequentes e Resolução, pex.

\section{Provadores, ATP e ITP}
Como computadores tem sido usados no processo de provas desse tipo.

\chapter{Introdução ao LEAN}
Capítulo sobre o provador utilizado no curso: o LEAN

\section{Teoria dos tipos}
Descrição superficial do teoria dos tipos para justificar o pardigma PAT.
Tambem justifica como o LEAN provas as coisas.

\section{Provas usando LEAN}
Exemplos (avancados) de provas utilizado LEAN.
