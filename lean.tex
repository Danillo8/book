\chapter{Introdução ao Lean}
% dessa vez baseado mais fortemente em
% https://leanprover.github.io/theorem_proving_in_lean/introduction.html
Ao final do último capítulo, definimos duas classes fundamentais de ferramentas de auxílio a prova de teoremas, os \textit{ATPs} e \textit{ITPs}.
Essas grupos de ferramentas em geral são muito bem definidos, e distanciados, mantendo um espaço vazio entre eles, o que pode parecer desvantajoso.
O matemático pode desejar algo entre a prova automática e interativa, ou o acesso a ambos os recursos numa mesma ferramenta.
Para isso serve o Lean.

O provador de teoremas Lean busca preencher essa lacuna: uma única ferramenta contendo o melhor de ambos os mundos acessíveis.
Nesse sentido, age como uma \textit{ponte entre os dois mundos}.

O ambiente do Lean suporta interação, construção de termos, e verificação passo-a-passo das expressões desenvolvidas.
Lean implementa o chamado \textit{calculus of constructions}, ou cálculo de contruções, um sistema dedutivo bastante poderoso, que contém os demais sistemas a serem discutidos ao longo do livro, tais como Lógica de Proposições, ou Lógica de Primeira Ordem.

\section{Teoria dos tipos}
Descrição superficial do teoria dos tipos para justificar o pardigma PAT.
Tambem justifica como o LEAN provas as coisas.

\subsection{Proposições como Tipos}
ou Terms as Proofs

\section{Provas usando LEAN}
Exemplos (avancados) de provas utilizado LEAN.

\subsection{Modo Termo}
Apresentação e exemplos

\subsection{Modo Tática}
Apresentação e exemplos

\subsection{Modo Cálculo}
Apresentação e exemplos

% será que faz sentido falarmos em aspectos praticos sobreo Lean, como onde ele está disponivel online, ou como conhecer os teormas basicos dispoiveis...
