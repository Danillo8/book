\section{Introdução}

\section{Regras de Inferência}
Quando nos comunicamos, é habitual utilizar diferentes estruturas de linguagem que nos aproximem do sentido que queremos dar a alguma sentença. Na lógica proposicional, esses padrões são também amplamente utilizados e importantes para a construção de fórmulas e serão melhor aprofundados a seguir.
%Sugestões
%Acho que poderíamos juntar a parte de semântica do capítulo 6 
%(na qual ele discute o que é algo ser verdadeiro ou falso e introduz as tabelas verdade) 
%com os capítulos anteriores (caps 2-5), 
%onde ele introduz as regras de inferência, a dedução natural e o Lean. 
%Então introduziríamos as regras, mostrando o seu equivalente na tabela verdade e também no Lean. 
%Depois dessa seção mostraríamos as árvores mais complexas de dedução natural tb dessa forma. 

\subsection{Implicação} 
A implicação é essencial para o condicionamento de sentenças. Se na lingua falada utilizamos a estrutura "Se ... então" para nos referirmos a acontecimentos que dependem de outros, na logica a ideia é muito semelhante.\\
Retornando aos primeiros problemas do capítulo, relembramos um dos problemas descritos:
\begin{center}
\textbf{Se Ana viajar para o Chile, conhecerá o Atacama}\\
\textbf{Se a família real não tivesse vindo ao Brasil, então o território se desintegraria}
\end{center}
Apesar das duas sentenças terem a mesma estrutura com "Se... então" podemos perceber que as duas tem suas diferenças. Em particular, chamamos a segunda proposição de implicação contrafactual, pois ela afirma como o mundo possivelmente seria caso as coisas fossem diferentes do que elas realmente são. Esse assunto é discutido por filósofos há seculos, Spinoza e Saul Kripke são nomes de destaque nesses estudos, entretanto, a lógica matemática se debruça mais especialmente na primeira sentença.
\subsection{Se e somente se}
%Vitoria

\subsection{Conjunção}
A regra da conjunção se refere ao uso ``e'' na linguagem informal, de forma que juntamos duas informações. Podemos ter frases como:
\begin{center}
\textbf{Santiago é a capital do Chile e o deserto do Atacama está localizado no Chile.}\\
\end{center}

Ao mesmo tempo, podemos utilizar o ``e'' para conectar duas informações que nem mesmo possuem relação entre si. Por exemplo, podemos dizer que:
\begin{center}
\textbf{Santiago é a capital do Chile e Bolsonaro é o presidente do Brasil.}\\
\end{center}
Quando utilizamos o ``e'', representado pelo símbolo $\land$ na lógica, o que de fato importa é estarmos unindo duas informações que são verdadeiras. Isso fica claro na tabela verdade abaixo, na qual é possível ver que quando $A$ ou quando $B$ são falsos, $A\land B$ é falso:

\begin{table}[htb]
\centering
\begin{tabular}{|l|l|l|}
\hline
\textbf{A} & \textbf{B} & \textbf{A $\land$ B} \\ \hline
V          & V          & V                  \\ \hline
V          & F          & F                  \\ \hline
F          & V          & F                  \\ \hline
F          & F          & F                  \\ \hline
\end{tabular}
\end{table}

Dessa forma, na dedução natural, podemos introduzir um ``e'', quando temos $A$ e também temos $B$. Assim, se $A$ e $B$ forem verdadeiros, $A \land B$ também vai ser. 

 \begin{prooftree}
     \AxiomC{A}
     \AxiomC{B}
     \BinaryInfC{$A \land B$}
\end{prooftree}

No Lean, representamos essa operação pela função $and.intro$, conforme o exemplo abaixo:
\vspace{5mm}
\begin{lstlisting} 
variables A B : Prop

example (h1 : A) (h2 : B) : A ∧ B :=
and.intro h1 h2
\end{lstlisting}
\vspace{5mm}

Seguindo esse raciocínio, é possível observar que sempre que temos $A \land B$, teremos $A$ e $B$ separadamente. Essa é a operação chamada de exclusão do ``e''.

Para excluir o $B$ de, por exemplo, $A \land B$, temos a chamada exclusão pela esquerda, conforme descrita abaixo:

 \begin{prooftree}
     \AxiomC{$A \land B$}
     \UnaryInfC{A}
\end{prooftree}

No Lean, essa operação pode ser feita utilizando a função $and.left$, conforme o código a seguir: 
\vspace{5mm}
\begin{lstlisting} 
variables A B : Prop

example (h1 : A ∧ B): A :=
and.left h1
\end{lstlisting}
\vspace{5mm}

Alternativamente, é possível realizar a mesma operação da seguinte forma:
\vspace{5mm}
\begin{lstlisting} 
variables A B : Prop

example (h1 : A ∧ B): A :=
h1.left
\end{lstlisting}
\vspace{5mm}
Para excluir o $A$, temos a chamada exclusão pela direita:

 \begin{prooftree}
     \AxiomC{$A \land B$}
     \UnaryInfC{B}
\end{prooftree}

No Lean, essa operação é realizada utilizando a função $and.right$, de maneira semelhante ao que foi descrito anteriormente para o $and.left$. 

\subsection{Disjunção}

Introdução do ``ou'':
\begin{prooftree}
     \AxiomC{A}
     \UnaryInfC{$A \lor B$}
\end{prooftree}

Código Lean de introdução do ``ou'':
\begin{lstlisting} 
variables A B : Prop

example (h1 : A): A ∨ B :=
or.inl h1
\end{lstlisting} 

Exclusão do ``ou'': 
%melhorar prova
%como adicionar "3 pontinhos" entre o A e o C na árvore da exclusão do "ou"?


\begin{prooftree}
    \AxiomC{$A \lor B$}
    \AxiomC{$A$}
    \UnaryInfC{$C$}
    \AxiomC{$B$}
    \UnaryInfC{$C$}
    \TrinaryInfC{$C$}
\end{prooftree}
     
Código Lean de exclusão do ``ou'':
%adicionar código

Tabela verdade do ``ou'':


\begin{table}[htb]
\begin{tabular}{|l|l|l|}
\hline
\textbf{A} & \textbf{B} & \textbf{A$\lor$B} \\ \hline
V          & V          & V                 \\ \hline
V          & F          & V                 \\ \hline
F          & V          & V                 \\ \hline
F          & F          & F                 \\ \hline
\end{tabular}
\end{table}

\subsection{Negação}
%Cristiana
\subsection{Prova por contradição}
%Cristiana

\section{Dedução Natural}

\section{Exercícios}
