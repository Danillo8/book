\section{Introdução}

\section{Regras de Inferência}
%Sugestões
%Acho que poderíamos juntar a parte de semântica do capítulo 6 (na qual ele discute o que é algo ser verdadeiro ou falso e introduz as tabelas verdade) com os capítulos anteriores (caps 2-5), onde ele introduz as regras de inferência, a dedução natural e o Lean. Então introduziríamos as regras, mostrando o seu equivalente na tabela verdade e também no Lean. Depois dessa seção mostraríamos as árvores mais complexas de dedução natural tb dessa forma. 

\subsection{Implicação} 
%Vitoria
\subsection{Se e somente se}
%Vitoria

\subsection{Conjunção}

Introdução do "e":
 \begin{prooftree}
     \AxiomC{A}
     \AxiomC{B}
     \BinaryInfC{$A \land B$}
\end{prooftree}


Exclusão do "e":
 \begin{prooftree}
     \AxiomC{$A \land B$}
     \UnaryInfC{A}
\end{prooftree}

\subsection{Disjunção}

Introdução do "ou":
\begin{prooftree}
     \AxiomC{A}
     \UnaryInfC{$A \lor B$}
\end{prooftree}

Exclusão do "ou": 

  \begin{prooftree}
        \AxiomC{$A \lor B$}
              \AxiomC{$A$}
              \UnaryInfC{$C$}
                   \AxiomC{$B$}
                   \UnaryInfC{$C$}
              \TrinaryInfC{$C$}
     \end{prooftree}

%como adicionar "3 pontinhos" entre o A e o C na árvore da exclusão do "ou"?

\subsection{Negação}
%Cristiana
\subsection{Prova por contradição}
%Cristiana

\section{Dedução Natural}

\section{Exercícios}
