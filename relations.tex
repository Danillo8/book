\chapter{Relações}
Em capítulos anteriores, discutimos proposições que lidavam com a relação entre objetos matemáticos. Muitas vezes na matemática, e mesmo no contexto em que estamos inseridos, estamos interessados em definir e estudar relações entre objetos distintos. Por exemplo, podemos estar interessados em certas própriedades sobre a relação \textit{é vais velho que}, entre seres vivos, e diremos que essa é uma relação \textit{irreflexiva}, \textit{transitiva}, ou ainda, uma relação de \textit{ordem estrita}.

Nesse capítulo discutimos exatamente essas noções, e definimos certos tipos de relaçõs mais comuns.

\section{Semântica das Relações}
Podemos abstrair a noção semântica de relação para um universo $R$ de tuplas de aridade definida, que contem a sequencia dos objetos relacionados. Por exemplo, considere que o elemento $a$ está relacionado a $b$ por uma relação $R$. Ppodemos denotar $aRb$, ou $R(a,b)$ significando que o par $(a,b)\in R$, está definido como existente no universo daquela relação. Como se pode esperar, a relação pode ter qualquer aridade necessária, e relacionar objetos de tipos distintos.

Considere, a partir disso, o universo de objetos $u = \{Ana, Bia, Cid\}$, e a relação \textit{conhece}, definida por $U\times U \supseteq A = \{(Ana, Bia), (Bia, Cid)\}$. Podemos dizer que \textit{Bia conhece Ana}?

%\section{Definições}
\theoremstyle{definition}
\newtheorem{definition}{Definição}[section]

%\theoremstyle{definition}
%\newtheorem{example}{Exemplo}[section]

\theoremstyle{plain}
\newtheorem{theorem}{Proposição}[section]

%\theoremstyle{plain}
%\newtheorem{corollary}{Corolário}[section]

Definimos uma série de tipos de relações importantes, frequentemente encontradas na literatura.
Note que muitas das definições se aplicam a relações conhecidas como \textit{maior que}, nos naturais, ou \textit{pertence} para conjuntos.

\section{Relações de Ordem}
Discutimos uma classe de relações binárias importantes: as chamadas relações de ordem.
Aqui, definimos relações de\textit{parciais} ou \textit{estritas}.
Usaremos os símbolos $\leq$ e $<$ para nos referir a relações quaisquer entre elementos de alguma estrutura $A$, e os usamos infixados: $x \leq y$ ou $x < y$.

\begin{definition}
    \label{partial_order}
    Seja $\leq$ uma relação. Dizemos que $\leq$ é de \textit{ordem parcial} se respeita as seguintes propriedades:

    \begin{itemize}
        \item \textbf{reflexividade:} para todo $x \in A$, $x\leq x$.
        \item \textbf{transitividade:} para todo $xy,z \in A$, se $x\leq y$, e $yleq z$, então $x\leq z$.
        \item \textbf{antissimetria:} para todo $x,y \in A$, se $x\leq y$ e $y \leq x$, então $x=y$.
    \end{itemize}
\end{definition}

\noindent Note que se entendemos $\leq$ por um predicado binário, as definições acima são facilmente expressos em lógica de primeira ordem.
Exemplos desse tipo são: $\leq$ em $\mathbb{N}$, $\mathbb{Z}$, $\mathbb{Q}$, e $\mathbb{R}$ ou a inclusão $\supseteq$ para a classe dos conjuntos.

Há ainda uma classe especial de relações de ordem vistas a seguir:

\begin{definition}
    \label{partial_order_total}
    Dizemos que a relação de \textit{ordem parcial} $\leq$ é \textit{total} se:

    \begin{itemize}
        \item para todo $x, y \in A$, $x\leq y$ ou $y\leq x$.
    \end{itemize}
\end{definition}

\noindent Vale observar que nos exemplos anteriores, apenas $\leq$ é total.
De fato, tome $A = \mathcal{P}(\mathbb{N}) $, os conjuntos $x={3}$ e $y={5}$ subconjuntos de $A$; claramente não vale a completude de $\subseteq$ em $A$. % Ainda, as relações \textit{divide}, $x|y$, nos inteiros, e outras...

O que dizer, no entanto, das relações \textit{menor} ou \textit{pertence}? De fato, essas pertencem a classe a seguir, as chamadas relações de \textit{ordem estrita}:

\begin{definition}
    \label{estrict_order}
    Considere $<$ relação em um conjunto $A$. Dizemos que a relação é de \textit{ordem estrita} $\leq$ é \textit{total} se:

    \begin{itemize}
        \item \textbf{transitividade: } para todo $x,y,z \in A$, se $x<y$ e $y<z$ então $x<z$.
        \item \textbf{irreflexividade: } para todo $x\in A$, $x\nless x$.
    \end{itemize}
    dizemos, ainda, que essa relação estrita é total em $A$ se:
    \begin{itemize}
        \item \textbf{tricotomia:} para todo $x,y \in A$, vale $x<y$, $x>y$ ou $x=y$.
    \end{itemize}
\end{definition}

\noindent Novamente, é facil ver como formalizar essas noções utilizando proposições em lógica de primiera ordem.

% uma relação estrita é assimétrica: prova.

A seguir, discutimos um resultado intuitivo que estabelece uma ligação importante entre as relações de ordem \textit{parciais} e \textit{estritas}:

\begin{theorem}
    \label{estrict_by_partial}
    Considere $\leq$ parcial em $A$. Podemos definir uma relação estrita $<$ em $A$, em que $x<y$ significa que $x\leq y$ e $x \neq y$. Ainda, se $\leq $ for total, então $<$ também será total.
\end{theorem}
\begin{theorem}
    \label{partial_by_estrict}
    Considere $<$ estrita em $A$. Podemos definir a relação de ordem parcial $\leq $ em $A$, em que $x\leq y$ significa que $x < y$ ou $x = y$. Ainda, se $<$ for total, então $\leq$ também será total.
\end{theorem}

\begin{proof}
    Exercício para o leitor!
\end{proof}

\section{Relações de Equivalência}
%Descrevemos as propriedades que definem esse tipo de relação, e damos exemplos. Mostramos as notações $a\sim b $, $a\equiv b$. Sugiro as relações de equivalencia "paralelo a", "modulo n", "mesma idade".


\subsection{Equivalencia e Igualdade}
Apenas discutimos brevemente como e porque Equivalencia e Igualdade são animais completamente diferentes. Toma os exemplos acima pra discutir.

\section{Relações em Lean}

\section{Exercícios}
