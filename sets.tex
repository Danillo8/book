\chapter{Conjuntos}

\section{Introdução}

    Para tratar de conjuntos iremos tentar uma abordagem que que não deixe de lado o rigor matemático necessário, mas que ao mesmo tempo permita utilizar o Lean e contextualizar exemplos do dia a dia. 
    
    Caro leitor, como você definiria conjunto? Vamos, pense um pouco. No século XIX, o matemático Georg Cantor, no jornal acadêmico Mathematische Annalen, descreveu um conjunto (ou utilizando sua terminologia, \textit{Menge}) como, em tradução livre: “Por conjunto nós entendemos qualquer coleção $M$ de objetos deteminados e distintos (chamados de elementos de $M$) da nossa intuição ou do nosso pensamento em um todo."
    
    Ou seja, mesmo um conjunto podendo ser algo tão abstrato quanto os números naturais ($\mathbb{N}$), os números reais ($\mathbb{R}$) e o Conjuntor de Cantor, podemos ter coisas menos abstratas como o conjunto das palavras desse texto, o dos planetas do Sistema Solar ou até dos alunos de sua turma. A questão é, todos esses conjuntos podem ser mais intuitivos ou abstratos (além de como são definidos) dependendo da pessoa que irá interpretá-los, por exemplo o conjunto dos naturais pode ser definido como $\mathbb{N} = \{1,2,3,...\}$ ou $\mathbb{N}=\{0,1,2,3,...\}$ (para não desagradar ninguém), o planetas do Sistema Solar $ P = \{ Mercurio, Venus, Terra, \\ Marte, Jupiter, Saturno, Urano, Netuno \}$ (lembramos que se este livro fosse escrito a uns 15 anos Plutão, na época, ainda seria considerado como um planeta, isto é, estaria no conjunto), por consequência, vemos que a interpretação do que determinado conjunto representa varia de pessoa para pessoa, mesmo que a ideia principal continue a mesma.
    
    Nosso intuito, durante essa aventura pelo mundo dos conjuntos será entender melhor certos conceitos e definições (como o fato do conjunto vazio estar contido em todos os conjuntos, ou se chegarmos até lá, porque o intervalo [0,1] nos reais não é enumerável), através de demosntrações, exemplos e exercícios, aumentando sua capacidade de abstração e a nossa também, já que para escrever esse capítulo nós teremos que ir além, pois não queremos apenas entender o que está aqui, mas que você entenda e aprenda também.

\section{Fundamentações}
    \subsection{Notações}
    Nesta seção iremos começar a introduzir notações matemáticas e algumas definições sobre conjuntos.
    
    Quando um determinado elemento $x$ faz parte de determinado conjunto $A$, nós dizemos que $x$ pertence a $A$ (denotamos $x \in A$). Caso $x$ não faça parte de $A$, diz-se que $x$ não pertence a $A$ (denota-se  $x \notin A$).
    
    Já quando um conjunto $B$ possui todos os elementos que $A$ possui e, $B$ tem, pelo menos, um objeto que $A$ não possua, dizemos que $A$ está contido em $B$ ($\space A \subset B$) ou que $B$ contém $A$ ($B \supset A$). Se não sabemos se $B$ possui um objeto que $A$ não possua (e $B$ ainda possui todos os elementos que $A$), denotamos $A \subseteq B$, que quer dizer que $A \subset B$ ou $A=B$, de modo equivalente $B \supseteq A$, significa que $B \supset A$ ou $A=B$. Se $A$ possui pelo menos um elemento que $B$ não possua, dizemos que $A$ não está contido em $B$ ($A \not\subset B$) ou que $B$ não contém $B\not\supset A$, assim $A \neq B$. 
    
    Se estamos interessados em conjuntos/elementos que pertencem simultaneamente a dois conjuntos $A$ e $B$, dizemos que estamos interessados na interseção de $A$ e $B$ (denotada como $A \cap B$).
    
    Já se estamos interessados nos conjuntos/elementos que fazem parte de $A$ ou de $B$ dizemos que, nosso objetivo é a união de $A$ e $B$ ($A \cap B$).
    
    \textbf{Definição de Universo :} Quando estamos trabalhando com conjuntos é comum definirmos quem é nosso universo ($ \mathcal U $), isto é, o conjunto que conterá todos os conjuntos/elementos que estaremos trabalhando em um contexto. Por exemplo, na reta real nosso universo é $\mathcal U = \mathbb{R}$.
    
    \textbf{Definição de Conjunto Complementar :} Sendo $A$ um conjunto, dizemos que o conjunto $A$ complementar (denotado como $\overline A$ ou $A^C$)contém todos os conjuntos/elementos que não estão contidos/pertencem a $A$, mas fazem parte de nosso universo ($U$).
    
    Lembre-se: os diagramas apresentados servem apenas para ajudar a entender os conceitos, mas não devem ser vistos como únicos, nem devem ser o foco único para entender o que é proposto.
    
    \subsection{Axiomas}
    
    Agora iremos apresentar alguns axiomas que servirão como base para todo o desenvolvimento dos conteúdos aqui propostos.
    
    \textbf{Axioma da Completude :} Dois conjuntos são iguais, se e somente se, todo elemento que pertence ao primeiro conjunto pertence ao segundo e, todo elemento que pertence ao segundo também pertence ao primeiro, ou seja:
    
    \[\forall A, \space \forall B, \space (A=B) \iff (\forall x,  (x \in A \iff x \in B))\]
